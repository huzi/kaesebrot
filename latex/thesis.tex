\documentclass [11pt,smallheadings, a4paper]{report}
\pagestyle{headings}

\usepackage[ngerman]{babel}

\usepackage{ucs}
\usepackage[utf8x]{inputenc}

\usepackage{capt-of} 
%\usepackage[style=numeric-comp]{biblatex}

%\usepackage{nomencl}
\usepackage{graphicx}
\usepackage{url}
\usepackage{amssymb}

\usepackage{color}
\usepackage{listings}

\usepackage{algorithm}
%\usepackage{algorithmic} 
\usepackage{program}

\usepackage{enumitem}
\usepackage{footnote} 

\lstset{language=Java,                % choose the language of the code
basicstyle=\footnotesize,       % the size of the fonts that are used for the code
numbers=left,                   % where to put the line-numbers
numberstyle=\footnotesize,      % the size of the fonts that are used for the line-numbers
stepnumber=1,                   % the step between two line-numbers. If it is 1 each line will be numbered
numbersep=5pt,                  % how far the line-numbers are from the code
backgroundcolor=\color{white},  % choose the background color. You must add \usepackage{color}
showspaces=false,               % show spaces adding particular underscores
showstringspaces=false,         % underline spaces within strings
showtabs=false,                 % show tabs within strings adding particular underscores
frame=single,   		% adds a frame around the code
tabsize=2,  		% sets default tabsize to 2 spaces
captionpos=b,   		% sets the caption-position to bottom
breaklines=true,    	% sets automatic line breaking
breakatwhitespace=false,    % sets if automatic breaks should only happen at whitespace
escapeinside={\%}{)}          % if you want to add a comment within your code
}
  \lstdefinelanguage{JavaScript}{
     keywords={attributes, class, classend, do, empty, endif, endwhile, fail, function, functionend, if, implements, in, inherit, inout, not, of, operations, out, return, set, then, types, while, use},
     ndkeywords={},
     sensitive=false,
     comment=[l]{//}
  }


\usepackage[pdftex,pdfborder={0 0 0}]{hyperref}

%\usepackage[left=3.5cm,right=3.5cm,top=2cm,bottom=2cm,includeheadfoot]{geometry}
%


\begin{document}

\newenvironment{myquote}%
{\begin{quote}\small}%
{\end{quote}}%

\pagestyle{empty}
%\noindent

    	\begin{figure}[hp]
				\centering
				\includegraphics[height=20mm, clip]{figures/inso-farbe}
		 	    \hspace{1mm}
		 	    \includegraphics[height=20mm, clip]{figures/info-farbe}
		 	    \hspace{1mm}
					\includegraphics[height=20mm, clip]{figures/tu-dt-teil-schw-pos}
				\label{fig:TU-Wien}
			\end{figure} 
			
\begin{center}
    \Large
    Research Group Industrial Software (INSO)\\
    Institute for Computer Aided Automation (E183)\\
    Faculty of Informatics\\
    Vienna University of Technology\\

\end{center}

\bigskip

\begin{center}
    \Large
    \centering Master of Science Thesis
    \end{center}
    
\bigskip

\begin{center}
    \Huge\bfseries
    Eine Domain Specific Language zur Berechnung und Evaluierung von Formularfeldern
\end{center}

\bigskip
\bigskip

\begin{center}
    \Large\bfseries
    \noindent
    \centering Author: \\Thomas Lemm\'{e} \\
    Beatrixgasse 25/1/12, 1030 Wien
\end{center}
\bigskip
\begin{center}
    \Large\bfseries
    \noindent
    \centering Supervisor: \\Thomas Grechenig \\
\end{center}
\bigskip
\begin{center}
    \Large\bfseries
    \noindent
    \centering Vienna, \today
\end{center}

\vspace*{\fill}

\cleardoublepage

\rmfamily
\normalfont


\noindent
\emph{Samma felxibl.}
\begin{flushright}
Klaus Feichtinger
\end{flushright}
% *************** Table of contents ***************
\pagenumbering{roman}
\pagestyle{headings}
\renewcommand{\baselinestretch}{0.9}\normalsize
\tableofcontents

% *************** End of front matter ***************

\pagestyle{headings}



\parindent0em
\parskip0.5em

%Zeilenumbrüche vermeiden
\sloppy 

\renewcommand{\baselinestretch}{1.0}\normalsize

\chapter*{Abstract}
Die vorliegende Arbeit beschäftigt sich mit der Modellierung von Ab\-hän\-gig\-kei\-ten in einem medizinischen Studiensystem. 

The fields of a web-form, as well as the underlying data, shall be put in relation using formulas and constraints. When entering data to a field, computations that set the values of dependent fields will be triggered.

The approach used in this master thesis is to design and develop a new expression language called Form Expression Language (FXL). The FXL is a domain-specific language (DSL) in the Domain ``relations in form-based data''. Similar to other expression languages, like formulas in spreadsheed software, FXL-statements are able to use variables to reference other fields and functions to extend the functionality of the language.

The first part discusses the technical and theoretical fundamentals, needed for the understanding of this thesis. An analysis of therms and definition is followed by an overview of the scientific field of relations in webforms. The second part is dedicated to the design process of the language. First the motivation for the development of a new language and its advantages over existing solutions is argued. After that the phases of the software development process, analysis, design, implementation and test, get described. During this process the language and its interfaces have been developed from the requirements.
The third part is about the integration of the FXL in an existing system for medical studies. It describes the system and its workflow to illustrate the use of the FXL in the system. Diverse topics have to be considered: execution time and location, execution order and detection of cyclic dependencies.


\pagenumbering{arabic}

\chapter*{Kurzfassung}
Die vorliegende Arbeit beschäftigt sich mit der Modellierung von Ab\-hän\-gig\-kei\-ten in einem medizinischen Studiensystem. Die Formularfelder eines Web-Formulars sowie die dahinter liegenden Daten, sollen durch Formeln und Bedingungen miteinander in Beziehung gesetzt werden. Durch Eingabe von Werten in einzelne Felder werden Berechnungen ausgelöst, die wiederum den Wert anderer Felder bestimmen.

Der Ansatz, mit dem diesem Problem begegnet wird, ist der Entwurf und die Entwicklung der Ausdruckssprache Form Expression Language (FXL). Diese ist eine Domain Specific Language (DSL) in der Domäne ``Modellierung von Formulardaten''. Ähnlich zu bekannten Ausdruckssprachen, wie etwa den Formel-Ausdrücken aus Tabellenkalkulationssoftware, ist die FXL in der Lage, Statements mit Variablen (zum Referenzieren anderer For\-mu\-lar\-fel\-der) und Funktionen (zur Erweiterung der Funktionalität) zu evaluieren.

Der erste Teil erörtert die zum Verständnis der Arbeit notwendigen technischen und theoretischen Grundlagen. Es werden die Begriffe definiert und eine Über\-sicht über das wissenschaftliche Umfeld der Modellierung von Webformularen geboten.
Der zweite Teil widmet sich dem Entwicklungsprozess der Sprache. Zuerst wird der Aufwand der Entwicklung einer neuen Sprache begründet, indem die Vor- und Nachteile der Alternativen abgewogen werden. Danach werden die Entwicklungsphasen Analyse, Entwurf, Implementierung und Test beschrieben. Im Zuge dieser Phasen wird aus den Anforderungen die Sprache und deren Schnittstellen entwickelt.
Der dritte Teil behandelt die Integration der in dieser Arbeit entwickelten DSL in das medizinische Studiensystem. Nach einer Beschreibung des Zielsystems werden diverse Themen wie Zyklenfreiheit, Ausführungsreihenfolge, Ausführungszeit- und Ort, behandelt, die bei der Integration beachtet werden müssen.



% include chapters here

\chapter{Einleitung}

Die voriegende Arbeit bewegt sich im Umfeld einer medizinischen Dokumentationssoftware, die zur Eingabe und
Verwaltung von medizinischen Daten verwendet wird. Die Software ist eine Webanwendung, die
Krankenhäusern, Ärzten und anderen Spezialisten den gemeinsamen Zugriff auf Daten für medizinische
Studien ermöglicht. Dabei ist jedoch die Anonymität der Studienteilnehmer gesichert, da medizinische Daten
nur pseudonymisiert und getrennt von Patientendaten abgelegt werden.

Die Software bietet die Möglichkeit, für verschiedene medizinische Fachbereiche bzw. Anforderungen
individuelle Eingabemasken zu erstellen. Dafür können aus unterschiedlichen Formularfeld-Typen flexibel
benutzerdefinierte Formulare erstellt werden. Manche Felder unterliegen dabei Anforderungen, was den Typ
der eingegebenen Daten bzw. den Datenbereich betrifft. In dieser Arbeit soll eine Domain Specific Language (DSL) entwickelt werden, 
welche eine erweiterte Modellierung von Abhängigkeiten und Validierungen in den Eingabemasken ermöglicht.


\section{Motivation}

Die Notwendigkeit für eine generische Lösung zur Berechnung und Evaluierung von Formularelementen
ist begründet durch das Bedürfnis der Benutzer nach Formularen, die automatisch Berechnungen über
mehrere Formulare durchführen. Solche Formulare wurden bisher als eigenes Plugin realisiert, das statt dem Formular 
angezeigt wird. Das Plugin ist somit genau auf den konkreten Anwendungsfall zugeschnitten. Der Nachteil dieser
Lösung ist allerdings, dass für jede Änderung der Sourcecode des Plugins bearbeitet, und die Anwendung 
neu deployed werden muss. Ein weiterer Nachteil ist, das für jeden weiteren Anwendungsfall mit anderen Berechnungen
ein weiteres Plugin notwendig ist.

Daraus ist die Anforderung nach einer allgemeineren und flexibleren Lösung zur individuellen Modellierung von 
Berechnungen in Formularen entstanden.


\section{Problemstellung}

Um eine flexible Modellierung von Berechnungen in Formularen zu erreichen, ist es notwendig, einzelne Felder
automatisch durch arithmetische Operationen mit Werten aus anderen Feldern zu berechnen, wie dies aus diversen
Tabellenkalkulationsprogrammen bekannt ist. Dazu sind zwei verschiedene Arten von Statements notwendig:
Formeln, die einem Feld einen Wert zuweisen und Bedingungen, die den Wert eines Formularfeldes
überprüfen. Die Eingabe der Statements soll direkt bei der Erstellung eines Formulars in der Webanwendung
geschehen. So ist es beispielsweise zielführend, ein Feld für den Body Mass Index automatisch aus
Körpergröße und Gewicht berechnen zu lassen, anstatt es manuell auszufüllen und bei jeder Änderung der
Daten neu einzugeben.

Ein weiterer wichtiger Teil der Problemstellung ist die Einbettung der Berechnungen und Constraints in das
bestehende Softwaresystem. Da dieses System bereits in Verwendung ist, ist es eine besondere
Herausforderung, die in der Arbeit vorgenommene Implementierung der DSL in das Studiensystem zu integrieren,
ohne die bestehende Architektur oder das Datenmodell im Hinblick auf eine notwendige Migration
maßgeblich zu verändern.


\section{Zielsetzung}

Das Ziel der Diplomarbeit ist, eine Domain Specific Language (DSL) zu entwickeln. Dazu ist es notwendig, zu definieren, was eine DSL
ist und was in diesem konkreten Anwendungsfall die Domäne darstellt. Die DSL soll es ermöglichen,
Beziehungen zwischen Formularfeldern herzustellen. Dabei ist darauf zu achten, dass die Syntax einfach
und konsistent ist. Die DSL soll in ihrer Syntax der Schreibweise von bekannten Systemen (Taschenrechner,
Tabellenkalkulation, …) ähnlich sein, so dass diese für erfahrene Computernutzer gut erlernbar und
anwendbar ist.

Der Kern der Arbeit ist der Entwurf und die Entwicklung einer do\-mä\-nen\-spe\-zi\-fi\-schen Sprache und deren
Integration in das bestehende Studiensystem. Der Fokus liegt dabei auf der Funktionalität der
domänenspezifischen Sprache. Nicht Teil der Arbeit sind formularübergreifende Referenzierungen, welche
jedoch in Konzeption und Design berücksichtigt werden. 

Ein Schwerpunkt der Arbeit ist die Integration der DSL in die bestehende Webanwendung. Dabei wird darauf
geachtet, dass die Referenzierung der verschiedenen Formularfelder für Formeln, die Werte aus
mehreren Feldern benötigen, für den Endbenutzer einfach und nachvollziehbar erfolgt. Weiters wird eine
Möglichkeit geschaffen, die Statements der DSL direkt in der Webanwendung zu den
entsprechenden Formularfeldern einzugeben. Schließlich wird die Ausführung der DSL in die Applikation
integriert, das heißt, wenn Daten eingegeben werden, müssen die angegeben Formeln berechnet, bzw.
Bedingungen überprüft werden. Inwieweit die Ausführung client- oder serverseitig geschieht, wird auch im
Hinblick auf Usability und Antwortzeiten zu betrachtet.




\section{Struktur}

Die vorliegende Arbeit ist in vier Teile gegliedert.

Der erste Teil, \emph{\nameref{part_grundlagen}}, beschäftigt sich mit der Aufbereitung der für das Verständnis der Arbeit notwendigen fachlichen Themen. Zuerst wird versucht, eine passende Definition des Begriffs `Domain Specific Languages' zu finden (Kapitel \ref{chapter_dsl} \nameref{chapter_dsl}). Kapitel \ref{chapter_theoretische_grundlagen} gibt eine Übersicht über die theoretischen Grundlagen, die zum Verständnis der Arbeit beitragen. In Kapitel \ref{chapter_tools} werden verschiedene Werkzeuge für die Entwicklung vorgestellt und insbesondere auf den Parsergenerator ANTLR eingegangen. Abschließend wird in Kapitel \ref{related_work} das wissenschaftliche Umfeld aufbereitet.

Im zweiten Teil der Arbeit, \emph{\nameref{part_entwicklung}}, werden die Entwicklungsschritte der aus dieser Arbeit resultierenden DSL beschrieben. Zuerst wird in Kapitel \ref{chapter_entscheidung} die Entscheidung für die Entwicklung einer eigenen Sprache begründet und die Alternativen besprochen. Danach werden in der Analyse die Anforderungen an die Sprache erhoben (Kapitel \ref{chapter_analyse}). In Kapitel \ref{chapter_entwurf} wird die DSL beschrieben. Datentypen, semantische Regeln und weitere Features werden genau spezifiziert. Kapitel \ref{chapter_implementierung} beschreibt die Implementierungsdetails der entwickelten Sprache. Kapitel \ref{chapter_test} widmet sich dem Testing in Hinsicht auf die Implementierung.

Der dritte Teil widmet sich der \emph{\nameref{part_integration}}. In Kapitel \ref{chapter_systembeschreibung} wird das medizinische Studiensystem beschrieben, in das die DSL integriert wird. Dazu wird der Workflow erklärt, um dem Leser den Nutzen der DSL im Kontext des Zielsystems zu erläutern. Zusätzlich wird die Architektur sowie der verwendete Technologie-Stack des Zielsystems beschrieben.

Der vierte Teil, \emph{\nameref{part_ergebnis}}, widmet sich in Kapitel \ref{chapter_ausblick} einem Ausblick auf weitere Themen, die über den Umfang dieser Arbeit hinausgehen und Raum für weitere Arbeiten geben. Zuletzt wird die Arbeit in Kapitel \ref{chapter_zusammenfassung} zusammengefasst.







\part{Grundlagen}
\label{part_grundlagen}
Im folgenden Kapitel werden die Grundlagen von Domain Specific Languages aufgearbeitet. 
Zuerst wird versucht, eine genaue Definition f"ur den Begriff DSL zu finden. 
Anhand von Beispielen wird gezeigt, wie flexibel dieser Begriff verwendet werden kann. 
Danach werden die Grundlagen von formalen Sprachen herausgearbeitet, die f"ur das Verst"andnis der Arbeit von Bedeutung sind. 
Ein weiteres Kapitel widmet sich den verschiedenen Tools, die das Erstellen von Sprachen unterstützen können und geht insbesondere auf das Werkzeug der Wahl, ANTLR, ein.
Das Ende des ersten Teils dieser Arbeit widmet sich dem Wissenschaftlichen Umfeld und gibt einen Überblick über weitere Arbeiten und Ansätze.



\newpage

\chapter{Domain Specific Languages}
\label{chapter_dsl}

Dieser Abschnitt beschreibt, was eine DSL ist, vergleicht die Vor-, Nachteile und Unterschiede zu General Purpose Languages (GPL), versucht eine für die vorliegende Arbeit geeignete Definition zu finden und gibt Beispiele für bekannte DSLs.

\section{Definitionen}

Dem Begriff Domain Specific Language steht der der General Purpose Language gegenüber. Mit GPLs lassen sich viele unterschiedliche Problemstellungen lösen, die nicht auf eine Branche oder einen Einsatzbereich beschränkt sind. Bekannte Programmiersprachen, die der Kategorie der GPLs zugerechnet werden, sind beispielsweise Java, C/++/\#, Python etc. Mit Domäne, oder auch Anwendungsdomäne wird ein abgegrenzter Einsatzbereich bezeichnet, der für diesen charakteristische, bzw. eben domänenspezifische, Anforderungen an ein System stellt.

Eine genaue Definition für den Begriff DSL zu finden ist nicht einfach, da die Bezeichnung in der wissenschaftlichen Community im Moment sehr inflationär gebraucht wird\footnote{Bevor sich der Begriff DSL etablierte, wurden domänenspezifische Sprachen oft als ``little languages'' bezeichnet., \cite{VaDe00}}. Eine kompakte Definition einer DSL gibt Martin Fowler in \cite{Fowl05}:

\begin{myquote}
Domain-specific language (noun): a computer programming language of limited expressiveness focused on a particular domain.
\end{myquote}

Eine sehr ähnliche, aber detailliertere Definition kommt von Mernik et al. \cite{MeHe05}:

\begin{myquote}
Domain-specific languages (DSLs) are languages tailored to a specific application domain. They offer substantial gains in expressiveness and ease of use compared with general-purpose programming languages in their domain of application.
\end{myquote}


Ein Unterschied, der ins Auge sticht, ist, dass Fowler die eingeschränkte Ausdrucksstärke ("limited expressiveness") betont, Mernik et al. aber den erheblichen Gewinn an Ausdrucksstärke ("'substantial gains in expressiveness"). Fowler meint damit, dass man durch die Fokussierung auf eine Domain eine Einschränkung des Einsatzbereiches erzwingt. Mernik et al. sehen den Gewinn an Ausdrucksstärke darin, dass mit einem einzelnen Statement in einer DSL mehr ausgesagt werden kann, als mit einem Statement einer GPL, das DSL-Statement ist also ausdrucksstärker.

Die zweite Definition geht im Unterschied zur ersten auf zwei essentielle Eigenschaften ein: Erstens muss eine DSL genau auf eine bestimmte Domain zugeschnitten sein und zweitens muss sie im Gegensatz zu einer GPL im spezifischen Einsatzbereich einfacher anzuwenden sein.

Ein Grund, warum Fowler in seiner Definition auf die Einfachheit im Gegensatz zu einer allgemeinen Programmiersprache verzichtet, mag seine Einteilung in interne und externe DSLs \cite{FoPa10} sein\footnote{Die Klassifikation in interne und externe DSLs ist nicht die einzige. So können sie auch nach Features oder Darstellungsart (textuell, tabellarisch, graphisch etc.) eingeteilt werden. \cite{LaJi07} }:

\subsection{Interne DSL}

Interne DSLs benutzen die Syntax und die Laufzeitumbgebung einer GPL. Oft wird unter der Bezeichnung interner (bzw. embedded) DSL eine entsprechende API für eine spezifische Domäne verstanden\footnote{Oft wird in diesem Zusammenhang Bjarne Stroustrup, der Entwickler von C++, zitiert: "Library design is language design"'}.
Das Problem ist das Mapping des domainspezifischen Problems auf eine GPL. Eine sehr beliebte Möglichkeit, APIs eine gewisse ``sprachähnlichkeit'' zu verabreichen, ist das sogenannte Method Chaining
\cite{RuBa08}.
Bei dieser Technik wird das Objekt, auf dem eine Methode aufgerufen wird, von ebendieser Methode wieder zurückgegeben. So können Methodenaufrufe verkettet werden, was sehr intuitive Statements zur Folge haben kann.

In manchen Sprachen können Sprachfeatures verwendet werden, um die Sprache zu einer eigenen DSL zu erweitern. Die Sprache JavaScript bietet etwa die Möglichkeit, Typen und Objekte zur Laufzeit dynamisch zu verändern (Listing \ref{listing_javascript_each}).\\

\begin{lstlisting}[language=JavaScript, caption={Erweiterung des Array-Typs um die Funktion \texttt{each()} in der JavaScript Bibliothek Prototype},label=listing_javascript_each]
['1', '2', '3'].each(function(elem){
	alert('Number ' +elem);
});
\end{lstlisting}

Eine weitere Möglichkeit, eine interne DSL zu verwenden, bietet die Einbindung einer Laufzeitumgebung einer allgemeinen Programmiersprache in ein Softwaresystem. Eine Methode bietet die Java Plattform mit der Einbettung von verschiedenen Script-Engines(vgl. Abschnitt \ref{section_java_scripting}).

\subsection{Externe DSL} 

Da externe DSLs nicht eingebettet in einer anderen Sprache ausgeführt werden, ist eine eigene Laufzeitumbebung notwendig. 

Eine externe DSL ist als individuelle Sprache mit eigener Syntax definiert, die allerdings auch existierende Notationen verwenden kann. Zusätzlich zur frei definierten Syntax, muss auch die Semantik explizit definiert werden. Der große Unterschied zu internen DSLs ist, dass die Komplexität der Sprache genau auf die Domäne, für die sie verwendet wird, beschränkt werden kann. Die Erstellung von Interpretern und Übersetzern wird durch eine Vielzahl meist freier Werkzeuge unterstützt.

In der Entscheidungsphase beim Design der DSL zur Berechnung und Evaluierung von Formularfeldern wird nochmals genauer 
auf das Thema interne vs. externe DSLs und deren Vor- und Nachteile im Bezug auf die Aufgabenstellung dieser 
Arbeit eingegangen (Kapitel \ref{chapter_entscheidung}).

Weiters führt Fowler auch noch \textit{Language Workbenches} als eigene Kategorie domänenspezifischer Sprachen auf. 
Darunter versteht man hochintegrierte Entwicklungsumgebungen, die nicht nur die Definition und Generierung von DSLs ermöglichen, 
sondern auch eine Entwicklungsumgebung für die erstellte Sprache zur Verfügung stellen
\footnote{Eine der bekanntesten Language Workbenches ist Jetbrains's MPS. Eine Übersicht über Language Workbenches und eine 
Vergleichsmatrix findet sich unter http://www.languageworkbenches.net/index.php?title=LWC2011\_Comparison\_Matrix, 20.12.2011}.



\section{Vorteile und Nachteile}

Wie bereits erwähnt, können domänenspezifische Sprachen einen erheblichen Gewinn an Produktivität und Wartbarkeit mit sich bringen\cite{VaDe98}. DSLs sind für den Endbenutzer besser zu handhaben. Durch die domänenspezifische Notation und der damit einhergehenden besseren Les- und Schreibbarkeit ist es für Domänenexperten leichter, die Sprache zu verwenden. 

Ein weiterer großer Vorteil ist die Validierung und Optimierung der Sprache auf Domain-Level. Das bedeutet, dass dem Benutzer viel detaillierter Feedback über die eingegebenen Programme und eventuell auftretende Fehler gegeben werden kann, als es eine allgemeine Programmiersprache könnte\cite{VaDe00}.

DSLs haben nicht nur Vorteile, sondern auch Nachteile und Einschränkungen. Ein Nachteil ist der notwendige Entwicklungsaufwand, um die Sprache zu erstellen. Der Aufwand, der zuerst investiert werden muss, benötigt einen entsprechenden Zeitraum, um sich zu amortisieren.

Die domänenspezifische Notation kann sich auch als Nachteil erweisen, weil sie die Möglichkeiten dahingehend einschränkt, dass wirklich nur die dafür vorgesehenen Aufgaben erledigt werden können. Statt einer Erweiterung der Funktion wie in GPLs durch eigene Klassen und Funktionen ist bei einer DSL eine Änderung der Sprachdefinition und der Laufzeitumgebung notwendig.

Einen weiteren Nachteil kann der Verlust von Performance und Effizienz einer selbst erstellten DSL im Vergleich zu einer ausgereiften GPL darstellen\cite{VaDe00}.



\section{Beispiele}
\label{grundlagen_examples}

Es finden sich diverse Beispiele für Sprachen, die auf einen eigenen Anwendungsbereich zugeschnitten sind. Nachfolgend werden einige Beispiele vorgestellt, die sich in der Informatik etabliert haben.

\paragraph*{SQL}
(Structured Query Language) ist eine Sprache zur Abfrage und Manipulation von Daten in relationalen Datenbanken. Sie eignet sich sehr gut als Beispiel für eine DSL, da sie die wesentlichen Eigenschaften hat: eine eigene, genau spezifizierte Domäne (Definition, Abfrage, Manipulation von Daten(banken)) und eine eingeschränkte Syntax, die auf die von GPLs bekannten Konstrukte wie Schleifen verzichtet.

\paragraph*{\LaTeX} ist ein Softwarepaket zum setzen von Texten, das auf der Software \TeX \ aufbaut. Im Gegensatz zu herkömmlichen Textverarbeitungswerkzeugen werden Texte in Latex nicht in einem graphischen Editor bearbeitet, sondern in einem einfachen Textdokument. Elemente wie Überschriften oder Bilder werden im Text mit einfachen Befehlen ausgezeichnet. Das Latex-Dokument kann dann in verschiedene Formate, wie Postscript, PDF oder HTML ausgegeben werden. Wie HTML is Latex also keine Programmiersprache, sondern eine Auszeichnungssprache.


\paragraph*{EBNF}
(Extended Bacus-Naur Form) ist - wie der Name vermuten lässt - eine Erweiterung der Bacus-Naur Form. Sie ist eine Metasprache zur Definition von Grammatiken und erlaubt die Darstellung von Produktionsregeln, Nichtterminalsymbolen und Terminalsymbolen in einer leicht verständlichen Form\cite{NiWi77}. Sie definiert Wiederholungen in geschweiften, und optionale Elemente in eckigen Klammern. Alternativen werden durch einen senkrechten Strich getrennt. 

\begin{lstlisting}[float = htbp,caption={Beschreibung eines Fließkomma-Typs in EBNF.},label=listing_example_ebnf]
DECIMAL = {DIGIT} "." DIGIT {DIGIT} | DIGIT {DIGIT} "." {DIGIT};

DIGIT   = "0"|"1"|"2"|"3"|"4"|"5"|"6"|"7"|"8"|"9" ;
\end{lstlisting}

Listing \ref{listing_example_ebnf} Beschreibt einen Fließkomma-Typ DECIMAL mit Hilfe der EBNF. Dieser Typ besteht aus beliebig vielen Ziffern vor und nach dem Kommapunkt, wobei jedoch zu beachten ist, dass mindestens eine Ziffer vor oder nach dem Kommapunkt verpflichtend ist.

Viele Tools zur Generierung von Parsern verwenden eine Abwandlung oder Erweiterung der EBNF zur Beschreibung der Grammatik (vgl. Abschnitt \ref{tools_vergleich}).




\chapter{Theoretische Grundlagen}

\label{chapter_theoretische_grundlagen}

In diesem Kapitel werden die theoretischen Grundlagen, die zum Verständnis der Arbeit notwendig sind aufbereitet. Diese beinhalten zunächst eine Übersicht über die anerkannten Methoden des Übersetzerbaus. Danach folgt eine Einführung in Formale Sprachen, die durch bestimmte Grammatiken erzeugt werden. Das ist vor allem deshalb notwendig, um ein Verständnis für die Arbeitsweise von Parsern zu erlangen. Weiters wird auf tiefere Konzepte wie Variablen und Funktionsaufrufe sowie den Umgang mit verschiedenen Datentypen eingegangen.


\section{Language Applications}
\label{theorie_language_applications}

Terence Parr definiert Language Applications nicht nur als Implementierung einer Sprache, sondern ``any program that processes, analyzes, or translates an input file''\cite{Parr10}. Diese allgemeine Definition ist vor allem für DSLs interessant, da diese of mehr  einer Konfigurationssprache als einer richtigen Programmiersprache ähnlich sind. Für das Auswerten von derart eingeschränkten Sprachen ist somit auch nicht unbedingt der volle Umfang eines Parsers, Interpreters oder Übersetzers im klassischen Sinne notwendig. Im Kontext dieser Arbeit wird nur ein Teil dieser Definition betrachtet, nämlich die Interpretation (im Sinne von Auswertung) von Ausdrücken, in Hinsicht auf erweiterte Konzepte von allgemeinen Language Applications. Darunter fallen in erster Hinsicht Variablen, Funktionen und Datentypen, die in der DSL verwendet werden sollen.

Der klassische Aufbau einer Language Application, wie er auch in diversen Lehrbüchern und Skripten zum Thema Compilerbau beschrieben ist, besteht aus den 4 Phasen lexikalische Analyse, syntaktische Analyse, semantische Analyse, Codeerzeugung, wobei im Falle einer interpretierten Sprache an Stelle der Codeerzeugung ein Laufzeitsystem tritt(Abbildung \ref{abb_language_application_klassisch})\footnote{Bei manchen interpretierten Sprachen wird zuerst Zwischencode erzeugt, der dann vom Interpreter abgearbeitet wird. Auch der AST (Abbildung \ref{abb_ast_vs_cst}) kann als Zwischencode betrachtet werden.}.


\begin{figure}[h]
\includegraphics[width=\textwidth,scale=0.5]{figures/language_application_klassisch.png}
\caption{Klassischer Aufbau einer Language Application: Compiler (linker Zweig) und Interpreter (rechter Zweig)}
\label{abb_language_application_klassisch}
\end{figure}

In der Lexikalischen Analyse wird das Programm eingelesen und in Tokens\footnote{Oft wird in der deutschsprachigen Literatur das Wort Symbol synonym verwendet.} aufgeteilt. Ein Token ist ein Paar bestehend aus einem Bezeichner oder Typ und einem optionalen Wert\cite[S. 111ff]{AhSe86}. Tokens können Keywords und Operatoren einer Sprache, Bezeichner von Funktionen und Variablen oder Literale sein. Im Zuge der lexikalischen Analyse können auch Zeichen wie Tabulatoren oder Leerzeichen, sofern diese für die Weiterverarbeitung nicht interessant sind, verworfen werden. Die Ausgabe nach dieser Phase ist eine Folge von Tokens, die von der syntaktischen Analyse für das Parsing verwendet wird.

Beim Parsing (Syntaktische Analyse) wird versucht, die Tokenfolge den Regeln der zugrundeliegenden Grammatik unterzuordnen. So kann überprüft werden, ob die Eingabe der Syntax der Sprache entspricht. Verschiedene Strategien des Parsens werden in Abschnitt \ref{theorie_parsing_strategien} beschrieben. 

Das Ergebnis der Syntaktischen Analyse kann in einem Parse Tree dargestellt werden. Dabei wird zwischen den Formaten Concrete Syntax Tree(CST) und Abstract Syntax Tree(AST) unterschieden (Abbildung \ref{abb_ast_vs_cst}). Der CST gibt die der Syntax der Sprache entsprechende Hierarchie der Tokens, die bei der Lexikalischen Analyse identifiziert werden, wieder. Für die Darstellung des AST weren jene Tokens eleminiert, die durch die Struktur des Baumes ihre Bedeutung verlieren\footnote{Im Beispiel in der Abbildung werden etwa die Klammern vor und nach der Addition obsolet, da diese in einem eigenen Subtree dargestellt wird.}.  

\begin{figure}[h]
\includegraphics[width=\textwidth,scale=0.5]{figures/ast_vs_cst.png}
\caption{Vergleich der Parse Tree Formate a) Concrete Syntax Tree(CST) und b) Abstract Syntax Tree(AST) für den Ausdruck x=(a+b)*c}
\label{abb_ast_vs_cst}
\end{figure}

Die Semantische Analyse überprüft Anforderungen an das Programm, die nicht durch die Syntax abgedeckt werden können. Das betrifft vor allem die Überprüfung, ob ein Programm dem Typsystem entspricht. Das Typsystem setzt sich zusammen aus der Definition der vorhandenen Datentypen und den Regeln zur Zuweisung der Typen zu Operatoren (vgl. \cite[S. 426]{AhSe86}). In vielen Programmiersprachen sind einige Operatoren überladen, d.h. sie können mit verschiedenen Datentypen verwendet werden. In der Phase der semantischen Analyse werden die Rückgabetypen aus dem Kontext abgeleitet. Die Typen von Variablen und Funktionen können aus einer Symboltabelle ausgelesen werden, in der Deklarationen derselben gesammelt werden.

Je nach Einsatzzweck wird die Eingabe dann in Maschinencode übersetzt oder in einer Laufzeitumbebung ausgeführt. Beim Compiler ist zu beachten dass die Codeerzeugungsphase in die Phasen Zwischencodeerzeugung, Codeoptimierung und Codeerzeugung aufgeteilt werden kann. 

Um den Rahmen der Arbeit nicht zu sprengen möchte ich hier zum tieferen Verständnis der einzelnen Phasen auf das Standardwerk \emph{Compilers: principles, techniques and tools} \cite{AhSe86} verweisen.


\section{Grammatiken}

Um die Syntax einer Sprache festzulegen ist eine formale Grammatik notwendig. Eine formale Grammatik $G \ = (\Sigma,\ N\ ,P\ ,S\ )$ besteht aus einer Menge von Terminalsymbolen $\Sigma$, einer Menge von Nichtterminalsymbolen $N$, den Produktionsregeln $P$, sowie dem Startsymbol $S$\footnote{Eine übersichtliche Einführung in die verschiedenen Grammatiken, sowie deren Unterschiede, Anwendungen und Normalisierungen, findet man in \cite{VoWi02}.}.

Eine Grammatik beschreibt die Menge aller syntaktisch korrekten Programme einer Sprache. Diverse Sprachen in unterschiedlichen Anwendungen erfordern unterschiedliche Typen von Grammatiken. Eine beliebte Einteilung wurde von Noam Chomsky in der sogenannten Chomsky-Hierarchie vorgenommen:

\begin{myquote}DEFINITION 6. For i = 1, 2, 3, a type i grammar is one meeting restriction i, and a type i language is one with a type i grammar. A type 0 grammar (language) is one that is unrestricted. 
Type 0 grammars are essentially Turing machines; type 3 grammars, finite automata. Type 1 and 2 grammars can be interpreted as systems of phrase structure description. \cite{Chom59}
\end{myquote}

Chomsky definiert also 4 Typen von Grammatiken, die unterschiedlich restriktive Ableitungsregeln haben:

\begin{itemize}
  \item Typ-0 Grammatiken: Dieser Typ repräsentiert die Menge aller formalen Grammatiken.
  \item Typ-1 Grammatiken: Grammatiken vom Typ 1 werden kontextsensitive Grammatiken genannt.
  \item Typ-2 Grammatiken: Dieser Typ definiert die kontextfreien Grammatiken. Die Syntax der meisten Programmiersprachen kann durch eine kontextfreie Grammatik beschrieben werden.
  \item Typ-3 Grammatiken: Diese sind jene Grammatiken, die durch reguläre Ausfrücke beschrieben werden können.
\end{itemize}

Der für Programmiersprachen relevante Typ ist also die Typ-2 bzw. kontextfreie Grammatik. Diese zeichnen sich dadurch aus, dass auf der linken Seite jeder Produktionsregel genau ein Nichtterminalsymbol steht (also ``frei von Kontext''). Auf der rechten Seite kann jede mögliche Folge von Terminal- bzw. Nichtterminalsymbolen stehen.



Meistens wird die Grammatik einer Sprache in einer Form dargestellt, die der Erweiterten Backus-Naur Form (EBNF, siehe Abschnitt \ref{grundlagen_examples}) ähnlich ist. In dieser Form sind die verschiedenen Bestandteile der Grammatik der Sprache (Terminal- bzw. Nichtterminalsymbole, Startsymbol und Produktionsregeln) gut ersichtlich und einfach nachzuvollziehen.

Je nach Parsingalgorithmus unterliegen die Grammatiken bestimmten Regeln, um vom Parser verarbeitet werden zu können.


\section{Parsing Strategien}
\label{theorie_parsing_strategien}

Ein Parser ist, allgemein gesprochen, ein Programm oder eine Softwarekomponente, die eine Eingabe in ein weiter verarbeitbares Format umwandelt. Es gibt verschiedene Ansätze, wie diese Aufgabe erfüllt werden kann. Grund\-sätz\-lich ist es möglich, eine Eingabe Top-Down oder Bottom-Up zu verarbeiten.

\subsection{Top-Down Analyse}

Bei der Top-Down Analyse wird versucht, ausgehend vom Startsymbol Ab\-lei\-tun\-gen zu finden, mit denen das gegebene Wort eingelesen werden kann. Dazu muss eine Tabelle erstellt werden, die jedem Eingabesymbol die ent\-sprech\-en\-de Produktionsregel zuordnet. Die Eingabe wird dann von links nach rechts mittels Linksableitungen abgearbeitet. Die Menge der Grammatiken, die mit dieser Methode verarbeitet werden können, werden LL(k) Grammatiken genannt. Das \emph{k} steht für den Lookahead, der notwedig ist, wenn einem Eingabesymbol mehrere Regeln in der Tabelle zugeordnet werden.


\subsection{Bottom-Up Analyse}

Die Bottom-Up Analyse geht den umgekehrten Weg, von der Tokenfolge zum Startsymbol. Sie versucht, in einer Eingabe eine Struktur zu finden, die den syntaktischen Regeln der Sprache entspricht. Nach und nach werden, abhängig vom Lookahead, ein oder mehrere Symbole eingelesen und davon ausgehend eine Produktionsregel aufgelöst.


\subsection{Erweiterte Konzepte und Optimierungen}
\label{theorie_erweiterte_konzepte}

\subsubsection{Backtracking} 

Unter Backtracking versteht man das Ausprobieren verschiedener Alternativen bei mehrdeutigen Produktionsregeln. Wenn eine Ableitung fehlschlägt, wird die Eingabe bis zur letzten Kreuzung zurückgespult und die nächste Alternative versucht. Erst wenn alle Alternativen erschöpft sind, wird ein Fehler zurückgegeben. Backtracking kann sehr aufwendig sein, da einzelne Regeln für eine Eingabe öfters aufgerufen werden. Um diese Mehrfachaufrufe zu reduzieren, kann eine Technik namens Memoization eingesetzt werden.

Zu beachten ist, dass unter Umständen Actions im Parser wiederholt oder umsonst ausgeführt werden, da Regeln aufgerufen werden, die im Endeffekt gar nicht benötigt werden.

\subsubsection{Memoization}

Memoization ist eine allgemeine Optimierungstechnik, um zu verweiden, dass Funktionen, die das gleiche Resultat zurückgeben,  wiederholt aufgerufen werden. Im Kontext des Parsing bedeutet dies, dass das Resultat von Ab\-lei\-tungs\-re\-geln zwischengespeichert wird. Wenn eine Regel aufgerufen wird, kann zuerst in der Tabelle nachgesehen werden, ob bereits ein Resultat vorliegt. Wenn dies der Fall ist, muss die Regel nicht nocheinmal ausgeführt werden.

Memoization erfordert zwar einen höheren Speicheraufwand beim Parsen, beschleunigt aber das Verfahren, da der Lookup wesentlich effizienter ist, als das wiederholte Parsen von Teilausdrücken.

\section{Typisierung}

Die meisten Programmiersprachen definieren verschiedene Datentypen, bzw. erlauben die Definition eigener Typen. Die Typisierung kann dabei statisch oder dynamisch erfolgen. Statische Typsysteme haben den Vorteil, dass sie bereits bei der Übersetzung überprüft werden können. Sie erhöhen somit die Wahrscheinlichkeit, dass das eingegebe Programm korrekt ist. Auch die Effizienz kann gesteigert werden, da eine Überprüfung zur Laufzeit nicht mehr notwendig ist. Bei dynamischen Typsystemen erfolgt die Zuweisung von Typen zu Variablen und Funktionen, und somit auch deren Überprüfung, erst zur Laufzeit des Programmes.

Da die Syntax einer Sprache im Großen und Ganzen unabhängig von der Typisierung ist, ist vor allem die Semantische Analyse für die Typüberprüfung zuständig. Lässt die Sprache die Deklaration von Variablen und Funktionen zu, muss deren (Rückgabe-) Typ in einer Symboltabelle abgespeichert werden. Wird die Variable oder Funktion in einem Statement des Programmes verwendet, muss der entsprechende Typ aus der Symboltabelle abgerufen werden.

Ein Teil des Typsystemes, ist die Zuordnung von Typen zu Operatoren. Da die meisten Operatoren überladen sind, müssen die tatsächlichen Typen während der Typüberprüfung aus dem Kontext abgeleitet werden. Bei Funktionsaufrufen ist zusätzlich eine implizite Typumwandlung notwendig. Das bedeutet, dass ein Typ in einen anderen umgewandelt werden kann, wenn dies keine semantische Änderungen zur Folge hat. Ein Beispiel ist eine Funktion mit einem Parameter vom Typ Gleitkommazahl. Ohne den Sinn der Funktion zu verändern, kann auch eine Ganzzahl an die Funktion übergeben werden, da die Menge der Ganzzahlen -- mathematisch gesehen -- eine Teilmenge der Gleitkommazahlen ist.


\chapter{Werkzeugunterstützung}
\label{chapter_tools}

Natürlich wäre es auch möglich, den Parser zur Verarbeitung der DSL-Statements ''per Hand'' zu programmieren. Allerdings bietet sich die Verwendung von vorhandenen Werkzeugen natürlich an. Die Verwendung von Tools hat neben dem einfacheren Entwicklungsprozess auch positive Aus\-wir\-kun\-gen auf die Wartbarkeit der DSL:

\begin{myquote}
Our results suggest that DSL tools do indeed increase maintainability
of DSL implementations. Especially our code reduction measurements
corroborate the hypothesis. The use of a DSL tool does
not necessarily make the resulting implementation less complex in
a structural sense, but it does usually remove the need for boilerplate
structure. \cite{KlSt10}
\end{myquote}

Es existieren viele verschiedene Werkzeuge und Frameworks im  Java-Umfeld, die die Entwicklung von DSLs unterstützen. Diese unterscheiden sich allerdings stark in den gebotenen Möglichkeiten, aber auch in den Methoden, was sicherlich auch an der unscharfen Definition des Begriffs DSL liegt. Manche Tools bieten eine gute Integration in eine IDE oder eine eigene Entwicklungsumgebung, einige generieren sogar Editoren, um die entwickelte DSL anzuwenden, andere sind simple Kommandozeilenprogramme.

Für die in dieser Arbeit zu entwickelnde DSL sind vor allem Parsergeneratoren, auch Compiler-Compiler genannt, interessant. Diese Tools generieren aus einer Grammatik in einem bestimmten, je nach Tool unterschiedlichen, Format einen Lexer und einen Parser (oder eine Kombination aus beiden). Der Output des Parsers, zumeist ein AST, kann in weiterer Folge verwendet werden, um einen Interpreter oder Übersetzer zu erstellen.


\section{Übersicht}

Bei allen Tools, die in diesem Abschnitt erwähnt werden, handelt es sich um Parsergeneratoren. Allen gemeinsam ist, dass sie die Grammatik einer Sprache in einem anwendungsabhängigen Format einlesen und verarbeiten. Das Produkt eines Parsergenerators ist ein lauffähiger  Parser (bzw. zusätzlich ein Scanner\footnote{Bei manchen Tools ist der Scanner im Parser integriert. Allerdings kann ein Scanner als eigene Komponente verwendet werden, um Informationen über ein Programm auszulesen, für die nur die Tokens benötigt werden. Der Scanner wird bei FXL z.B. verwendet um alle Variablen einer Formel zu erhalten, indem die Tokens vom Typ VAR aus der Tokenfolge ausgelesen werden. Siehe auch Abschnitt \ref{theorie_language_applications} bzw. der Algorithmus zur Bestimmung der Ausführungsreihenfolge in Abschnitt \ref{implementierung_integration_reihenfolge}.}), der ein Programm in der zu entwickelnden Sprache in einen AST verwandelt (Abbildung \ref{abb_parser_generator_workflow}). 

Eine weitere Möglichkeit ist, Ausdrücke bereits beim Parsen auszuwerten. Die meisten Tools unterstützen Actions, die bei der Ausführung einer Ableitungsregel im Parser ausgeführt werden. Diese Möglickeit hat allerdings einige Nachteile:

\begin{itemize}
  \item Die Actions werden direkt in die Grammatikdatei geschrieben und machen so die eigentliche Grammatik schlecht lesbar.
  \item Die Statements der Actions werden in der Zielsprache des Parsers eingegeben. Editoren für Grammatiken unterstützen allerdings nicht den Umfang eines Codeeditors einer IDE, was die Entwicklung fehleranfälliger und schwieriger macht.
  \item Verwendet der Parser Backtracking, kann es sein, dass einige Actions mehrfach ausgeführt werden.
  \item Typechecks ohne Ausführung des Statements wären nur durch eine eigene Grammatik für diesen Zweck möglich, was zu Redundanz und erhöhter Fehleranfälligkeit führt.
\end{itemize}

Für diese Arbeit wurde entschieden, dem Prinzip der \emph{Separation of Concerns} zu folgen und den erzeugten AST in einem Typechecker bzw. Interpreter explizit abzuarbeiten.

 

\begin{figure}[h]
\includegraphics[scale=0.6]{figures/parser_generator_workflow}
\caption{Workflow eines Parser Generators (blau).}
\label{abb_parser_generator_workflow}
\end{figure}



\section{Vergleich}
\label{tools_vergleich}

Um eine Auswahl für die Implementierung des Interpreters zu treffen, ist es notwendig, die verschiedenen Tools anhand verschiedener Eigenschaften zu vergleichen. Das betrifft vor allem das Format der Grammatik und die vorhandenen Ein-/Ausgabesprachen.
Zusätzliche Faktoren sind die Werkzeugunterstützung (Grammatikeditor, Rule-Visualisierung, (Concrete/Abstract) Syntax Tree-Visualisierung), Dokumentation und (Community-)Support sowie Tools für den Build-Prozess.

In diesem Kapitel werden die Tools ANTLR, Coco/R und JavaCC verglichen. Diese sind alle für die Java-Plattform verfügbar und grundsätzlich für die in dieser Arbeit zu entwickelnde Sprache geeignet. Zusätzlich können diese Produkte als ausgereift betrachtet werden und wurden erfolgreich in verschiedenen Projekten eingesetzt.


\subsection{ANTLR}

Der Parsergenerator ANTLR wird von Terence Parr seit 1989 an der Universität von San Franzisco entwickelt und hat sich zu einem Quasi-Standard auf der Java Enterprise Plattform entwickelt. Unter anderem wird ANTLR im Applicationserver Weblogic, dem ORM-Mapper Hibernate, der IDE IntelliJ IDEA oder der JBoss Rules Engine Drools eingesetzt\footnote{vgl. http://www.antlr.org/showcase/list, 23.3.2012}.

ANTLR ist in Java implementiert, unterstützt aber diverse Zielsprachen wie Java, C++, C\#, Ruby, Objective C uvm. Für die Entwicklung steht eine ausgeprägte Palette an Tools zur Verfügung (siehe Abschnitt \ref{tools_antlr_tools}). Die Grammatik ist ein Format, das sich an der EBNF-Notation orientiert, aber erweiterte Konzepte wie Actions und Attribute zum Aufbau eines AST unterstützt. Lexikalische und Syntaktische Regeln können in seperaten Files abgelegt werden.

Neben der ausführlichen Dokumentation auf der Projektwebseite existiert auch ein Buch des Autors als Einfürung und Referenz\cite{Parr07}. Aufgrund der weiten Verbreitung in der Java-Welt steht ausreichend Support durch die Community zur Verfügung.



\subsection{Coco/R} 

Das Coco/R Projekt\cite{HaMo90} ist ein Parsergenerator, der ursprünglich von Hans\-peter Mössenböck an der Universität Linz entwickelt wurde. Zwischenzeitlich wurde das Projekt an der ETH Zürich fortgesetzt und ist inzwischen wieder an der Universität Linz beheimatet.

Coco/R ist nicht auf eine Plattform beschränkt und unterstützt diverse Sprachen. Es existieren Portierungen für u.A. Java, C++, C\# und VB.NET. Als Basis für die Angabe der Grammatik wird die EBNF-Notation verwendet. Angereichert wird diese durch Attribute und \emph{semantic Actions}. Durch diese Attribute kann ein AST aufgebaut werden\footnote{vgl. \cite{Moes11}}, allerdings muss dazu, im Gegensatz zu ANTLR, die ganze Grammatik abgeändert werden.

Auf der Homepage des Projkets steht ein ausführiches User Manual, sowie ausreichend Beispiele für jede Sprach-Portierung zum Download bereit. Zusätzlich bietet eine Mailingliste der Universität Linz Support. Es existiert ein Eclipse-Plugin für einen Grammatikeditor und die Erzeugung des Parsers beim Build.

\subsection{JavaCC}


JavaCC ist ein aus dem Sun-Projekt Jack entstandener Parsergenerator für die Java-Plattform. Auf der Projekthomepage wird er als ``the most popular parser generator for use with Java [tm] applications''\footnote{vgl. http://javacc.java.net/, 29.11.2011} bezeichnet. Eine Untermauerung für diese Behauptung, z.B. durch Projekte die JavaCC einsetzen, ist nicht zu finden. Laut Wikipedia\footnote{vgl. http://de.wikipedia.org/wiki/JavaCC, 29.11.2011} verwenden die Suchmaschine Lucene und das Ontologie-Framework Cyc durch JavaCC erstellte Parser.

Die Syntax wird in einer Grammatikdatei definiert, in der sowohl lexikalische als auch syntaktische Regeln angegeben werden. Die Ableitungsregeln enthalten zwei Blöcke: einen für die Ableitungsregel und einen für die Auswertung (sogenannte \emph{Action}) der Regel. Alternativ kann mithilfe des Tools JJTree ein Syntax-Baum erstellt werden.

Die Dokumentation ist recht ausführlich gestaltet. Für Support durch die Community steht eine Mailingliste zur Verfügung. Es gibt ein Eclipse-Plugin, das einen Code-Editor für JavaCC Grammatiken enthält. JavaCC ist in Java geschrieben und generiert Parser ausschließlich in Java.





\section{ANTLR}
\label{tools_antlr}
Dieser Abschnitt setzt sich im Detail mit ANTLR, dem Werkzeug der Wahl für diese Arbeit, auseinander. Die Entscheidung für die Form Expression Language (FXL) ANTLR einzusetzen, begründet sich durch die hervorragende Dokumentation, insbesondere dem Buch des Autors, dem guten IDE Support und der Integration in das Build-Tool Maven.


\subsection{Grammatik}

Wie in Abschnitt \ref{tools_vergleich} erwähnt, orientiert sich das Format der Metasprache des Parsergenerators an der EBNF. Zusätzlich werden zusätzliche Angaben und Optionen in der Grammatikdatei gespeichert. Als Beispiel einer kompletten Grammatik kann jene der FXL im Anhang betrachtet werden.

\subsubsection{Optionen \& zusätzliche Angaben}

Es ist möglich, die Erzeugung von Parser und Lexer durch diverse Einstellungen zu beeinflussen.

\paragraph{Options}

Am Anfang eines Grammarfiles steht eine Auflistung von Parametern in einem \texttt{options}-Block. Diese steuern grundlegende Eigenschaften des generierten Parsers:

\begin{itemize}
  \item Ausgabesprache: Wie in Abschnitt \ref{tools_vergleich} erwähnt, beherrscht ANTLR diverse Ausgabesprachen. Hier muss angegeben werden, in welcher Sprache Lexer und Parser generiert werden sollen.

  \item Ausgabeformat: Das Format, das vom Parser beim Parsen eines Eingabestrings zurückgegeben wird. Dies kann entweder ein AST oder ein StringTemplate\footnote{StringTemplate ist eine Template Engine für verschiedene Einsatzbereiche. http://www.stringtemplate.org/, 23.3.2012} Template sein. Wird die Option weggelassen, wird nichts generiert, der Parser arbeitet dann als Recognizer, der nur den Eingabestring verarbeitet und im Fehlerfall eine Fehlermeldung ausgibt.

  \item Weitere Optionen: Weitere Optionen umfassen vor allem optimierende Parsingstrategien wie Memoizing und Backtracking (siehe Abschnitt \ref{theorie_erweiterte_konzepte}). Eine genaue Auflistung der Optionen befindet sich in der Re\-fe\-renz\cite{Parr07}.
\end{itemize}



\paragraph{Tokens umbenennen}

Literale können einfach als Strings, eingeschlossen in einfachen Anführungszeichen, in der Grammatik verwendet werden (siehe Listing \ref{listing_grammar_tree_compare}). Um diese Literal-Tokens im Interpreter besser verwenden zu können, bietet es sich an, diese mit sprechenden Namen zu benennen. Durch die Angabe \texttt{OR = 'OR';} im \texttt{tokens}-Block, wird im Parser automatisch eine Konstante \texttt{Parser.OR} generiert, auf die beim Abarbeiten des AST zugegriffen werden kann.

Weiters können auch Tokens erstellt werden, die in den Ableitungsregeln gar nicht vorkommen, aber für den Aufbau des AST benötigt werden. Ein Beispiel ist der \texttt{CALL}-Token in Listing \ref{listing_grammar_tree_function}, der beim Rewrite des AST verwendet wird (siehe folgender Abschnitt).

\subsubsection{Modellierung des Abstract Syntax Trees}

Würde man zur Weiterverarbeitung nach dem Parsen den Concrete Syntax Tree (CST) verwenden, würden die zahlreichen Knoten einen erheblichen Mehraufwand bedeuten. Deshalb ist es wichtig, wie der AST, der dann im Interpreter verwendet wird, aufgebaut sein soll. Dazu bietet ANTLR die Möglichkeit, in der Grammatik zu spezifizieren, wie der Baum aussehen soll. Das kann auf zwei Arten geschehen: inline oder explizit als Rewrite-Regel.

Bei der Inline-Angabe (Listing \ref{listing_grammar_tree_compare}) wird nur jenes Element (Literal, Terminal, oder Nonterminalsymbol) ausgewählt, welches als Wurzelelement des AST-Knotens gewählt wird. Die übrigen Elemente werden in der Reihenfolge ihres Auftretens als Kindelemente angehängt. Elemente, die im AST ignoriert werden sollen, weil sie in der Baumdarstellung ihre Relevanz verlieren (z.B. Klammern, Trennzeichen etc.), werden mit einem Rufzeichen markiert und nicht als Kindelement zum Knoten hinzugefügt.


\begin{lstlisting}[float = htbp,caption={Inline-Regeln zur Beschreibung der Baumstruktur.},label=listing_grammar_tree_compare]
compareExpression
  	:
  	commonExpr (('<'|'>'|'='|'<='|'>='|'!=')^ commonExpr)?
  	;

\end{lstlisting}

Rewrite Regeln können die Baumstruktur auch stark verändern und Knoten einführen, die als solches nicht in der Regel vorkommen. Dazu wird die Regel wie in Listing \ref{listing_grammar_tree_function} einfach an die rechte Seite der Ableitungsregel angehängt. Nun kann der Knoten für diese Regel im Format \texttt{\textasciicircum(Root Child1 ... Childn)} angegeben werden, wobei Root den Tokentyp des AST-Knotens bezeichnet, gefolgt von den Kindknoten. 


\begin{lstlisting}[float = htbp,caption={Rewrite-Regeln zur Beschreibung der Baumstruktur.},label=listing_grammar_tree_function]
functionCall
  :
  ID '(' arguments ')' -> ^(CALL ID arguments?)
  ;
\end{lstlisting}

Im Beispiel in Listing \ref{listing_grammar_tree_function} wird der Tokentyp \texttt{CALL} als Typ des Knotens festgelegt. Dazu muss dieser Token allerdings zuerst im Token-Abschnitt des Grammarfiles festgelegt werden (siehe oben). Als erstes Kindelement wird ein Knoten vom Typ \texttt{ID} erstellt, der den Namen der Funktion enthält. Die weiteren Kindknoten sind die optionalen Argumente, wobei diese wiederum komplexe Ausdrücke in Form von Subtrees enthalten können (siehe FXL-Grammatik im Anhang).

\begin{figure}[h]
\includegraphics[scale=0.5]{figures/ast_beispiele}
\caption{a) Aufbau des AST bei Inline-Notation. b) Aufbau des AST bei Rewrite-Notation}
\label{abb_ast_beispiele}
\end{figure}

Abbildung \ref{abb_ast_beispiele} Zeigt jeweils ein Beispiel für die zwei Tree-Building Methoden. Die Knoten enthalten jeweils den Tokentyp und den Text des Knotens. In Beispiel a) wird das Kleiner-Zeichen als Wurzelelement verwendet, die anderen Elemente werden als Knoten vom Typ Integer als Kindknoten angehängt. Beispiel b) zeigt den Sonderfall des künstlichen Tokens \texttt{CALL} aus Listing \ref{listing_grammar_tree_function}, der nur zum Umformen des Baumes verwendet wird.


\subsection{Tools}
\label{tools_antlr_tools}

Werkzeuge können den gesamten Entwicklungsprozess effizienter und einfacher gestalten. Sei es durch Tools wie einem Editor oder Debugger bei der Programmierung, oder solche zur Automatisierung des Buildprozesses. Vor allem Integrierte Entwicklungsumgebungen (IDEs) unterstützen den Entwickler beim Design einer Sprache. Im Falle eines Parsergenerators besteht eine IDE aus einem guten Editor für die Grammatik (Syntax Highlighting, Code Completion etc.), visuellen Darstellungen der Grammatik und einzelner Regeln, sowie einer Testumgebung zur Evaluierung der Syntax.

\subsubsection{ANTLRWorks}

ANTLRWorks \footnote{vgl. http://www.antlr.org/works/index.html, 23.3.2012} ist ein Tool zum Entwickeln von Grammatiken für ANTLR. Es ist in vielerlei Hinsicht eine Hilfestellung, da es vor allem visuelle Hilfestellungen, wie Syntaxdiagramme oder Syntaxbäume, bietet. Der Hauptteil von ANTLRWorks besteht aus dem Grammatikeditor und der Auflistung der Grammatikregeln. Der Editor selbst bietet einige Features, die von anderen IDEs bekannt sind wie etwa Syntax Highlighting und Code Completion.

\begin{figure}[h]
\includegraphics[scale=0.35]{figures/antlrworks_syntax_diagram}
\caption{Syntax Diagram in ANTLRWorks}
\label{abb_antlrworks_syntax_diagram}
\end{figure}

Ein weiteres Feature ist die Darstellung der Regeln in Syntaxdiagrammen (Abbildung \ref{abb_antlrworks_syntax_diagram}). Syntaxdiagramme werden dazu verwendet, Grammatiken graphisch darzustellen. Insbesondere können einzelne Regeln, die textuell durch Verzweigungen und Wiederholungen recht schnell komplex werden, mit Hilfe von Syntaxdiagrammen recht einfach nachvollzogen werden.

\begin{figure}[h]
\includegraphics[scale=0.35]{figures/antlrworks_interpreter}
\caption{Interpreter/Concrete Syntax Tree in ANTLRWorks}
\label{abb_antlrworks_interpreter}
\end{figure}

ANTLRWorks kann auch beliebige Eingabestrings parsen und gegen die verschiedenen Ableitungsregeln testen. Der resultierende Parsebaum wird graphisch dargestellt. Tritt ein Fehler auf, so wird das Parsen unterbrochen und die Stelle im Parsebaum, an der der Parser nicht fortsetzen kann, markiert. Auf diese Art können nicht nur Statements die von der Startregel ausgehen getestet werden, sondern auch Eingaben für beliebige Unterregeln.


\begin{figure}[h]
\includegraphics[scale=0.35]{figures/antlrworks_debugger}
\caption{Der Debugger von ANTLRWorks}
\label{abb_antlrworks_debugger}
\end{figure}

Ein Key-Feature von ANTLRWorks ist der Debugger (Abbildung \ref{abb_antlrworks_debugger}). Mit Hilfe des Debuggers lässt sich der Parseprozess Schritt für Schritt nachvollziehen. Während des Parsens wird für jeden Schritt angezeigt, welche Regel gerade angewendet wird. Zusätzlich wird der Parsebaum im mittleren Fenster schrittweise aufgebaut.

ANTLRWorks beherrscht sogar das Debuggen von Parsern in anderen Sprachen. Dies ist möglich, da ANTLRWorks per Sockets mit dem Parser kommuniziert\footnote{ vgl. http://www.antlr.org/works/index.html, 23.3.2012}.


ANTLRWorks ist ein mächtiges Werkzeug, das den Entwickler effektiv bei der Entwicklung von Grammatiken unterstützt. Der Nachteil ist jedoch, dass es ein eigenständiges Tool ist, dass zusätzlich zu diversen anderen Tools in den Entwicklungsprozess eingebunden werden muss. Für eine bessere Integration in diesen stehen diverse IDE-Plugins zur Verfügung, die im folgenden Abschnitt behandelt werden.
 
\subsubsection{IDE-Plugins}

Mittlerweile existieren diverse Plugins für die wichtigsten Entwicklungsumgebungen\footnote{Eine aktuelle Übersicht über die Plugins für diverse IDEs findet sich unter http://www.antlr.org/wiki/display/ANTLR3/Integration+with+Development+Environments}. Für IntelliJ IDEA existiert ein Plugin, das ANTLRWorks direkt in die IDE integriert. Der Funktionsumfang ist dementsprechend gleich umfangreich wie ANTLRWorks selbst.

Für die weit verbreitete offene IDE Eclipse existieren zwei Plugins. AntlrDT stellt vor allem einen Code-Editor und die Integration in den Eclipse Buildprozess (automatisches Compilieren nach Clean bzw. Speichern) zur Verfügung. Zusätzlich bietet das Plugin auch die Visualisierung von ASTs an, die allerdings, im Vergleich zu ANTLRWorks, nicht so ausgereift erscheint. Das zweite Eclipse-Plugin ist ANTLR IDE\footnote{http://antlrv3ide.sourceforge.net/, 23.3.2012}. Es ist sehr ausgereift und wurde auch für die Implementierung der Form Expression Language in der vorliegenden Arbeit verwendet. Es bietet einen Grammatikeditor, eine Ansicht für Syntaxbäume (im Plugin Railroad View genannt) und einen Interpreter, mit dem wie in ANTLRWorks Eingaben auf verschiedene Regeln angewendet werden können. Auch ein Debugger ist vorhanden, der allerdings im Gegensatz zu ANTLRWorks nur mit Java-Parsern arbeiten kann.

\subsubsection{Build Tools} 

Alle oben genannten Plugins integrieren ANTLR in den Buildprozess der Entwicklungsumgebung. Für ANTLR bedeutet dies, dass Lexer und Parser automatisch beim Build der Software aus der Grammatik generiert wird. Oft wird allerdings ein externes Tool zum Build bzw. zur Auslieferung von Software verwendet. Für die zwei verbreitetsten Buildtools auf der Java-Plattform, Apache Ant und Apache Maven, existieren Erweiterungen, die die Codegenerierung übernehmen\footnote{vgl. http://www.antlr.org/wiki/display/ANTLR3/How+to+use+ant+with+ANTLR3 bzw. http://www.antlr.org/antlr3-maven-plugin/index.html}.



\chapter{Related Work}
\label{related_work}

In diesem Kapitel wird eine Übersicht über das wissenschaftliche Umfeld dieser Arbeit geboten. Einige der vorgestellten Ansätze und Projekte gaben Anregungen für die Form Expression Language, die in dieser Arbeit entwickelt wurde. Es lassen sich zwei Bereiche identifizieren, die dem Umfeld der Arbeit entsprechen. Einerseits verschiedene Sprachen zur Berechnung und Evaluierung von Werten sowie DSLs zur Modellierung von Webformularen. Die Themen Domain Specific Languages und Parser Generatoren wurden bereits weiter oben behandelt.

\section{Sprachen zur Berechnung und Evaluierung von Werten}

Eine Sprache, die sich in wissenschaftlichen Arbeiten oft wiederfindet, ist die Object Constraint Language (OCL)\cite{RiGo98}.Die Object Constraint Language (OCL) ist ein Teil der UML - Spezifikation und wurde ursprünglich für die Modellierung von Klassen- und Sequenzdiagrammen angedacht. Ein weiterer Einsatzort ist die Modellgetriebene Entwicklung. Dort wird OCL etwa im Bereich Transformation von (Meta-) Modellen und Modelchecking verwendet\footnote{vgl. http://www.eclipse.org/modeling/mdt/?project=ocl, 21.3.2012}. Escott et al.\cite{Esco12} setzen die OCL zur Validierung von Webformlaren ein (siehe unten).

Eine weitere Sprache, die einfache Berechnungen und Validierungen ermöglicht, ist die Unified Expression Language (UEL). Diese ist aus den sehr ähnlichen Expression Languages für JavaServer Pages(JSP) und JavaServer Faces(JSF) entstanden. Die UEL kann auf Java-Objekte zugreifen, Werte setzen beziehungsweise abfragen und Methoden aufrufen. Eine freie Implementierung ist JUEL\footnote{http://juel.sourceforge.net/, 21.3.2012}. JUEL hat den Vorteil, dass sie außerhalb des Java Server Kontexts für beliebige Anwendungsgebiete verwendet werden kann.

Auf der Java-Plattform entstanden verschiedene Implementierungen von Skriptsprachen, die innerhalb der Anwendung ausgeführt werden können. Zwei der bekanntesten Projekte sind Mozillas JavaScript-Engine Rhino\cite{wwwRhino} und die Python und Ruby Portierungen Jython bzw. JRuby. Der Java Specification Request JSR-233 definiert eine standardisierte API \cite{JSR-223}, um Skripte in einer Java Anwendung auszuführen und Java-Objekte an die Engine zu binden\footnote{vgl. Abschnitt \ref{section_java_scripting}}.  Seit Java 1.6 stehen einige Scriptengines out-of-the-box zur Verfügung, u.a. JavaScript, Groovy und Ruby. Programme in Form von Skripts können zur Runtime erstellt und verändert werden. Es können Variablen an die Engine übergeben und im ausführenden Skript verwendet werden. Zusätzlich kann auf die gesamte Java API zugegriffen werden. Nach der Ausführung kann wiederum auf die Variablen zugegriffen werden. 

Zuletzt sollen noch die Formelsprachen von Tabellenkalkulationswerkzeugen wie Microsoft Excel oder LibreOffice Calc Erwähnung finden, da diese ein Vorbild für die Syntax der FXL sind.


\section{DSLs zur Modellierung von Formularen}

Der Wunsch nach erweiterten Möglichkeiten zur Modellierung von Web Formularen ist nicht neu. Dieser Wunsch äußert sich in verschiedenen Ansätzen, Webanwendungen allgemein und Formulare im Speziellen semantisch anzureichern. Die verschiedenen Ansätze weisen unterschiedlichen Stärken und Schwächen auf. In diesem Abschnitt wird auf verschiedene Arbeiten in diesem Gebiet eingegangen.

Eine Schwierigkeit besteht darin, dass Web-Applikationen in eine Client- und eine Serverseite aufgeteilt sind, wobei der Client meistens nicht vertrauenswürdig ist. Anwendungen der Clientseite (vor allem mit HTML und JavaScript) können durch deren offene Natur leicht manipuliert werden. Berechnungen und Validierungen im Client können also nicht als sicher angesehen werden, eine zusätzliche Überprüfung auf dem Server ist also notwendig. 

Zwei Ansätze, die auf Kosten der Offenheit versuchen, die Manipulierbarkeit von Anwendungen in den Griff zu bekommen, sind Adobe Flash (bzw. Flex und Air) und Java Applets (bzw. JavaFX). Diese setzen allerdings, im Gegensatz zu den gängigen Technologien HTML und JavaScript, die mit verschiedenen Clients funktionieren, mehr oder weniger proprietäre Laufzeitumgebungen für die Applikationen vorraus.

Ein Ansatz für eine DSL zur Modellierung von Formularen, ist das Projekt Mawl der Bell Laboratories\cite{AtBa99}. Mawl ist eine DSL für ``Form-based services''mit der aus HTML-Templates und der Programmlogik in der DSL Webanwendungen entwickelt werden können. Die Applikation wird zu Skripts für die CGI Schnittstelle kompiliert. Als Nachfolger von Mawl gilt das Projekt Bigwig\footnote{http://www.brics.dk/bigwig/, 15.3.2012}, ein Framework zur ``Entwicklung von interaktiven Web Services''. Bigwig-Programme werden in C-Code, HTML und JavaScript umgewandelt. Das Framework bietet Komponenten für verschiedene Apekte wie dynamische Dokumente, Security, Datenbankanbindung und Validierung von Eingaben, PowerForms genannt. Der Ansatz von Powerforms\cite{BrMo00} ist eine deklarative DSL im XML-Format, um HTML-Formulare durch Constraints zu erweitern. Die Constraints können, wie in dieser Arbeit gefordert, auch andere Formularfelder referenzieren. Der Nachfolger JWIG ist als allgemeines Web-Application Framework zu betrachten, ohne explizite Features einer DSL.

Ein weiterer Ansatz, der erweiterte Modellierungen von Webformularen ermöglicht, ist XFORMS. XFORMS ist ein W3C-Standard\cite{Xfor09}, der die detaillierte Modellierung von Formularen zum Ziel hat. Formulare können clientseitig durch diverse Constraints, Validierungen und Berechnungen erweitert werden. Durch das \texttt{calculate} Attribut werden Berechnungen mit Werten anderer Felder ermöglicht\cite{Chas07}. XFORMS wäre Teil des XHTML 2.0 Standards gewesen, der sich allerdings gegen HTML 5 nicht durchsetzen konnte.

WebEAV\cite{NaPr00} ist ein Framework zur Metadaten-gesteuerten Generierung von Webformularen für Entity-Attribute-Value Datenbanken. Metadaten beschreiben, wie einzelne Formulare zusammengesetzt werden sollen. Es ist möglich, Constraints und Abhängigkeiten zu definieren, die auf der Clientseite berechnet werden. WebEAV generiert Scripts, die Abhängigkeiten zwischen den Feldern beschreiben. Die Berechnungen im Browser erfolgen mit der JavaScript Funktion \texttt{eval()}, die Programmcode als Script entgegennimmt und ausführt. Wie in dieser Arbeit können jene Felder, die von einer Berechnung beeinflusst werden, ebenfalls neu berechnet werden.

Escott et al.\cite{Esco12} entwerfen einen Anzatz für die modellgetriebene Entwicklung, in dem UML-Modelle mit zusätzlichen Informationen angereichert werden. Drei Typen von Validierungen werden identifiziert: \emph{Single element}, \emph{Multiple element} und \emph{Entity association}. \emph{Single element} bezeichnet Validierungen, die sich auf das Element selbst beziehen, also beispielsweise Beschränkung der Werte auf minimale oder maximale Schranken. \emph{Multiple element} bezieht sich auf Felder, die von anderen Feldern abhängig sind. Diese Constraints werden dem Modell als OCL-Ausdrücke hinzugefügt. Der dritte Typ, \emph{Entity associations}, validiert die Multiplizitäten des UML-Modells.

WebDSL\cite{Vis08} ist eine DSL für die Entwicklung von Webanwendungen. Die Elemente einer Applikation, etwa Datenmodell, User Interface, Benutzerinteraktionen und Zugriffskontrolle werden in einem sehr hohen Abstraktionsgrad beschrieben. Der DSL-Code wird in eine Java Webanwendung übersetzt und auf dem Servlet Container Tomcat deployed. Ein Teil der WebDSL ist die Integration der Datenvalidierung\cite{GrVi09} auf verschiedenen Ebenen: Überprüfen von Werten und Typen, Constraints des Models, Model-unabhängige Constraints, sowie Constraints beim Bearbeiten eines Formular-Requests. Für die letzten drei Ebenen können boolesche Ausdrücke erstellt werden, die auf Daten des Requests und des Models zugreifen können.




\part{Entwicklung der Form Expression Language - FXL}
\label{part_entwicklung}
\input{kapitel_design}

\part{Integration in ein existierendes System}
\label{part_integration}
Im dritten Teil der Arbeit wird die im zweiten Teil entwickelte DSL in eine bestehende medizinische Dokumentationssoftware eingebunden.

Zuerst wird das Studiensystem SPICS beschrieben, um dem Leser der Arbeit eine Übersicht über das System zu bieten. Es soll helfen, das not\-wen\-dige Verständnis für den Nutzen der FXL zu erzeugen. Danach werden die einzelnen Aspekte und Überlegungen bei der Integration vorgestellt.

\chapter{Systembeschreibung}
\label{chapter_systembeschreibung}

Dieses Kapitel erläutert den Aufbau der Dokumentationssoftware SPICS, um ein Verständnis für die Anwendung der Form Expression Language in der Implementierung herzustellen. Neben der Beschreibung des technischen Aufbaus werden auch die konkreten Anwendungsfälle der FXL erklärt, um nachvollziehen zu können, an welchen Stellen in den Workflow eingegriffen wird.


\section{SPICS}

SPICS (Secure Platform for Integrating Clinical Services) ist ein Projekt der  Forschungsgruppe Industrial Software der TU Wien. Die Software ist eine Webanwendung, die Krankenhäusern, Ärzten und anderen Spezialisten -- unter Berücksichtigung des Datenschutzgesetzes -- den gemeinsamen Zugriff auf Daten für medizinische Studien er\-mög\-licht. 

Die SPICS Plattform bietet die Möglichkeit, für verschiedene medizinische Fachbereiche bzw. Anforderungen individuelle Eingabemasken zu erstellen. Dafür können aus unterschiedlichen Formularfeldern flexibel benutzerdefinierte Formulare erstellt werden. Weitere Features sind der Import und Export der gesammelten Studiendaten, um diese mit externen Statistik-Tools auszuwerten. Dabei ist zu beachten, dass die Anonymität der Patienten und Studienteilnehmer gesichert ist, da die medizinischen Daten nur pseudonymisiert abgelegt werden.


\section{Workflow}
\label{section_systembeschreibung_workflow}

Um die Integration in das Dokumentationssystem zu verstehen, werden zuerst die wichtigsten Anwendungsfälle erörtert, um zu veranschaulichen, wo die DSL die Arbeit mit den Daten erleichtern kann. Zuerst wird der typische Ablauf anhand eines Beispiels durchgespielt, um danach zu erläutern, wo die FXL den Workflow verbessern kann.

SPICS enthält ein umfassendes Rollen- und Berechtigungssystem. Die für diese Arbeit essentiellen Rollen sind jene des Administrators und des Contributors. Administratoren sind berechtigt, Formulare zu erstellen und zu bearbeiten. Contributoren sind hingegen jene Benutzer, die Daten in die Formulare eingeben.

\subsection{Workflow anhand eines Beispiels}
\label{subsection_workflow_beispiel}


Um den typischen Ablauf zu verdeutlichen, werden die maßgeblichen Anwendungsfälle für ein beispielhaftes Formular angeführt. Im darauf folgenden Abschnitt wird eruiert, wo die FXL in diesen Ablauf eingreift um den Endanwender bei der Eingabe der medizinischen Daten zu unterstützen.

Natürlich enthält ein so aufwändiges System wie SPICS neben dem hier vorgestellten auch zahlreiche weitere Workflows, etwa für Datenexport und -import, Benutzerverwaltung, Konfiguration und Terminverwaltung. Auf diese soll hier jedoch nicht eingegangen werden, um den Focus auf das Thema der Arbeit nicht zu verlieren.

Der Workflow soll am Beispiel eines Formulars gurchgegangen werden, in dem die Veränderung des Body-Mass-Index (BMI) eines Patienten dokumentiert wird. Ein Arzt (bzw. aus Software-Sicht: der Endanwender) soll also in der Lage sein, den BMI eines Patienten an verschiedenen Tagen zu dokumentieren. Abbildung \ref{abb_workflow_formular_ausfuellen} auf Seite \pageref{abb_workflow_formular_ausfuellen} zeigt das fertige Formular, in das die Daten eingegeben werden können.

Das in den folgenden Schritten erstellte Formular soll im gesamten zweiten Teil der Arbeit als Beispiel dienen, da es einerseits einfach nachzuvollziehen ist und andererseits einige grund\-sätz\-liche Konzepte und Überlegungen beinhaltet, die im weiteren Verlauf der Arbeit erläutert werden\footnote{vgl. Abschnitt \ref{implementierung_dsl_eingabe}  \nameref{implementierung_dsl_eingabe} sowie Abschnitt \ref{implementierung_daten_eingabe} \nameref{implementierung_daten_eingabe}}.

\subsection{Use Cases}

\subsubsection{Erstellen eines neuen Formulars}

Der erste Schritt im Workflow ist das Erstellen eines neuen Formulars. Es ist zuerst ein Titel zu vergeben, sowie einzustellen, ob das Formular pro Patient einmal oder mehrmals ausfüllbar ist. Diese Unterscheidung ist not\-wen\-dig, da manche Daten nur einmal zu erheben sind (z.B. Stammdaten, oder Daten zur Geburt eines Patienten). Oft ist es allerdings auch not\-wen\-dig (wie im oben beschriebenen Beispiel), ein Formular für einen Patienten mehrmals auszufüllen, etwa um einen zeitlichen Verlauf zu dokumentieren.
 
\begin{figure}[h]
\begin{center}
\includegraphics[scale=0.5]{figures/workflow_neues_formular}
\caption{Erstellen eines neuen Formulars}

\label{abb_workflow_neues_formular}
\end{center}
\end{figure}

Zu\-sätz\-lich können auch weitere Parameter angegeben werden (Abbildung \ref{abb_workflow_neues_formular}), die beispielsweise das Layout oder die Sprache betreffen. Diese Einstellungen haben allerdings für das Beispielformular keine Relevanz.

Das Ergebnis dieses Schrittes ist ein leeres Formular, das nun mit einer beliebigen Anzahl und Anordnung von Formularelementen gefüllt werden kann.

\subsubsection{Editieren eines Formulars}

Wenn ein Formular neu erstellt wurde (bzw. wenn es bereits existiert oder importiert wurde) kann es editiert werden. Darunter versteht man das Anlegen, Verändern und Löschen von Formularelementen. Zu\-sätz\-lich werden Formularelemente in Gruppen zusammengefasst, um eine übersichtliche Struktur zu schaffen.

Manche Formularelemente können mit einfachen Constraints belegt werden. So kann bei einem Datumsfeld der Datumsbereich (von-bis) ausgewählt werden. Bei Textfeldern kann der Datentyp insofern festgelegt werden, als dass zwischen \emph{Text}, \emph{Ganzzahl} und \emph{Kommazahl} unterschieden wird. Bei den beiden letzteren ist wiederum ein Constraint möglich, der den Wertebereich einschränkt. Außerdem kann für jedes Formularelement festgelegt werden, ob es ein Pflichtfeld ist.

Für das Beispielformular wird zuerst eine Gruppe mit dem Namen `BMI-Rechner` erstellt. In dieser Gruppe werden nun drei Textfelder
für das Kör\-per\-ge\-wicht, die Kör\-per\-grö\-ße und den BMI angelegt. Zu\-sätz\-lich wird noch eine Checkbox mit dem Titel `Adipositas` hinzugefügt.

\subsubsection{Auswählen eines Patienten}

Der nächste Schritt ist die Auswahl eines Patienten, für den das Formular ausgefüllt werden soll. Hier findet ein Wechsel der Rolle des Benutzers vom Administrator zum Contributor statt. Waren für die bisherigen Schritte Berechtigungen zum Erstellen und Editieren von Formularen not\-wen\-dig, so ist ab nun lediglich die Berechtigung zum Ausfüllen der Formulare erforderlich. 

Zunächst wird ein Patient anhand seines Synonyms aus der Liste aller Patienten ausgewählt. Auf der Übersichtsseite für einen Patienten sind alle bereits ausgefüllten Formulare aufgelistet. Soll ein neues Formular hinzugefügt werden, muss der Name des Formulars ausgewählt werden. 

\subsubsection{Ausfüllen eines Formulars}

\begin{figure}[h]
\begin{center}
\includegraphics[scale=0.5]{figures/workflow_formular_ausfuellen}
\caption{Ausfüllen des neuen Formulars}

\label{abb_workflow_formular_ausfuellen}
\end{center}
\end{figure}

Wurde auf der Übersichtsseite eines Patienten das Formular `BMI-Rechner` ausgewählt, erscheint nun das in den vorigen Schritten erstellte Formular (Abbildung \ref{abb_workflow_formular_ausfuellen}). Der Endanwender kann nun die ent\-sprech\-enden Daten in die dafür vorgesehenen Felder eingeben. Der BMI muss hier mithilfe eines Taschenrechners oder Tools berechnet werden. Ist sein Wert größer 30, so handelt es sich laut WHO um Fettleibigkeit (Adipositas)\footnote{ vgl. \cite{Who00} }. In diesem Fall sollte die Checkbox `Adipositas` angeklickt werden.



\section{Architektur \& Technologien}

Dieser Abschnitt enthält einen Überblick über die Softwarearchitektur und die verwendeten Technologien, um die Integration der Form Expression Language aus technischer Sicht nachvollziehen zu können.


\subsection{Architektur}

Die Architektur von SPICS ist als klassische 3-Tier Applikation mit graphischer Benutzeroberfläche, Programmlogik und Datenzugriffsschicht ausgeführt (Abbildung \ref{abb_spics_architektur}). 

\begin{figure}[h]
\begin{center}
\includegraphics[scale=0.4]{figures/spics_architektur_neu}
\caption{Übersicht über die 3-Tier Architektur}

\label{abb_spics_architektur}
\end{center}
\end{figure}

Die Benutzerschnittstelle ist eine Webapplikation, die auf die Backing Beans der Logikschicht zugreift. Unterstützt wird die GUI durch Java\-Script-basierte Widgets wie z.B. Date Picker. Wo möglich bzw. sinnvoll werden AJAX-Calls verwendet, um den Workflow flüssiger zu gestalten.

Der Kern der Software besteht aus Komponenten, die die Programmlogik enthalten. Diese sind einerseits Backing Beans, deren Attribute an die Benutzerschnittstelle gebunden werden, andererseits Action Beans, in denen Daten verarbeitet werden. Zu\-sätz\-lich bietet die Software auch noch eine Webservice-Schnittstelle und Komponenten zum Import und Export von Formularen und medizinischen Daten.

Die Komponenten der Logik-Schicht greifen auf die darunter liegende Datenbank über Data Access Objects (DAOs). Das Datenmodell des Systems wird über objektrelationales Mapping auf die Tabellen der relationalen Datenbank abgebildet.



Eine zusätzliche Besonderheit von SPICS ist die Trennung der persönlichen Patientendaten von den Behandlungsdaten. Diese örtliche Trennung ist not\-wen\-dig, um den Datenschutz der Patienten zu gewährleisten. Das Zusammenführen der perönlichen Daten mit den medizinischen Daten erfolgt erst im Browser des Benutzers. Auf diese Eigenschaft wird hier nicht weiter eingegangen, da sie für die Aufgabenstellung nicht relevant ist. Sie soll aber trotzdem erwähnt werden, da dies ein Key-Feature von SPICS darstellt.

\subsection{Technology Mapping}

Der Kern der Dokumentationssoftware SPICS ist als Java Enterprise Applikation, basierend auf dem Framework Seam ausgeführt. Als Applicationserver wird der JBoss AS verwendet. Auf die darunterliegende PostgreSQL Datenbank wird mittels Hibernate als ORM Mapper zugegriffen.

Verwendete Technologien des Systems:

\begin{itemize}
	\item Java 6
	\item PostgreSQL
	\item JBoss AS 5.1.0.GA
	\item Seam 2.2
	\item JSF 1.2
\end{itemize}



\chapter{Integration}


Im folgenden Kapitel werden die einzelnen Themen der Integration der Form Expression Language in das Studiensystem SPICS präsentiert. Zuerst werden die Anwendungsfälle der FXL in SPICS beschrieben und dargestellt, wo diese die Anwendung und in weiterer Folge das Datenmodell beeinflussen. Danach werden die verschiedenen Aspekte bei der Modellierung von Formularen mit Hilfe der FXL, sowie die Eingabe der Daten in das Studiensystem beschrieben. Zuletzt wird noch auf die Implementierung der Fehlerbehandlung eingegangen, die dem Benutzer Feedback über Probleme und deren Ursachen durch genaue Fehlermeldungen bereitstellt.


\section{Die FXL in SPICS}

Wie der Titel der Arbeit aussagt, soll die Modellierung der Formulare durch die entwickelte domänenspezifische Sprache unterstützt werden. Das zu\-sätz\-li\-che Feature in Hinsicht auf die Modellierung ist, dass die Integration der FXL er\-mög\-licht, die Formularelemente eines individuellen Formulars miteinander in Beziehung zu setzen.

\subsection{Anwendungsfälle}

In der aktuellen Version sind die einzelnen Eingabefelder eines Formulars unabhängig voneinander. Eine Eingabe in einem Feld, kann den Wert eines anderen Feldes nicht beeinflussen. Der Endanwender muss dafür Sorge tragen, dass die Felder zueinander sinnvolle Werte beinhalten. Wenn der berechnete BMI im Beispielformular größer als 30 ist, sollte der Bearbeiter des Formulars die Adipositas-Checkbox anckecken. Es gibt allerdings in der aktuellen Version des Systems, abgesehen von speziellen Plugins, keine Möglichkeit zu überprüfen, ob die Ckeckbox korrekt gesetzt wurde.

Es kristallieren sich die folgenden zwei Anwendungsfälle heraus:

\begin{itemize}
  \item Berechnen von Formularfeldern mit \textbf{Formeln}.
  \item Überprüfen von Formularfeldern durch Bedingungen (\textbf{Constraints})\footnote{Die Begriffe Bedingung, Einschränkung und Constraint werden in weiterer Folge für diesem Context synonym verwendet.}
\end{itemize}

\paragraph{Formeln} 

Im BMI-Formular des Beispiels bieten sich zwei Felder für die automatische Berechnung durch eine Formel an. Einerseits natürlich der BMI, der sich aus den Feldern Körpergewicht und Kör\-per\-grö\-ße mit der Formel

\begin{center}
 BMI = Körpergewicht in kg / ( Kör\-per\-grö\-ße in m )$ ^2 $
\end{center}

berechnen lässt. Anderererseits lässt sich aber auch der Wert der Adipositas-Checkbox durch die Formel

\begin{center}
 Adipositas = BMI $ > $ 30
\end{center}

berechnen. Eine Checkbox kann nur zwei Zustände haben: ausgewählt und nicht ausgewählt. Diese Zustände stehen implizit für die booleschen Werte wahr und falsch. Wie in Tabelle \ref{tbl_semantische_typregeln} ersichtlich, geben Vergleichsoperatoren allgemein einen booleschen Wahrheitswert zurück. Der Rückgabewert der Adipositas-Formel kann also verwendet werden, um den Wert der Checkbox ent\-sprech\-end zu setzen.

Allgemein kann also gesagt werden, dass Formeln verwendet werden, um den Wert eines Formularelements (in Abhängigkeit zu anderen Formularelementen) zu verändern.

\paragraph{Constraints}

Im Beispielformular gibt es auch Einschränkungen, was die Gültigkeit bzw. Sinnhaftigkeit der Eingabegrößen betrifft. Sowohl die Kör\-per\-grö\-ße, als auch das Körpergewicht muss großer als null sein, da negative Werte natürlich nicht sinnvoll sind. Ein weiterer Constraint kann sein, dass ein Feld ausgefüllt sein muss und daher als Pflichtfeld definiert wird. Diese Einschränkungen hängen allerdings nicht von anderen Formularelementen ab und werden auch, wie in Abschnitt \ref{subsection_workflow_beispiel} erwähnt, vom momentanen System unterstützt.

Eine Erweiterung der Überprüfung wäre, wenn andere Formularfelder in die Überprüfung mit einbezogen werden könnten. Angenommen man möchte die Nachbehandlung einer Operation dokumentieren und erstellt ein Formular, in das zusätzlich zu den medizinischen Daten der Operationstermin, sowie der Termin der Nachbehandlung eingetragen wird. In diesem Fall ist es not\-wen\-dig, dass der Termin der Nachbehandlung erst nach dem Operationstermin stattfindet. Es handelt sich also um eine Bedingung, die vom Wert eines anderen Formularfeldes abhängig ist.

Es muss also eine Formel definiert werden, die den Constraint beschreibt. zusätzlich muss diese Formel einen booleschen Wahrheitswert als Rückgabetyp haben, da ja überprüft werden soll, ob der Wert eines Feldes gültig ist (true) oder nicht (false). Für das obige Beispiel könnte die Formel wie folgt aussehen:

\begin{center}
 Constraint = Nachbehandlung $ > $ Operationstermin
\end{center}

Allgemein kann also gesagt werden, dass Constraints verwendet werden, um den Wert eines Formularelements (in Abhängigkeit zu anderen Formularelementen) zu überprüfen.


\subsection{Anknüpfungspunkte}

Da nun der Zweck der Form Expression Language in der konkreten Anwendung klar ist, stellt sich die Frage, wo nun im oben angeführten Workflow Än\-der\-ung\-en vorgenommen werden müssen, um die Funktionalität der DSL in die Anwendung zu integrieren. Es ist klar, dass die Statements der DSL einerseits eingegeben\footnote{vgl. Abschnitt \ref{implementierung_dsl_eingabe}  \nameref{implementierung_dsl_eingabe}} und andererseits ausgeführt\footnote{vgl. Abschnitt \ref{implementierung_daten_eingabe} \nameref{implementierung_daten_eingabe}} werden müssen. 

Die Statements der FXL werden bei der Modellierung des Formulars eingegeben. Hier erfolgt in erster Linie die Syntaxüberprüfung und die semantische Analyse mit der ent\-sprech\-enden Fehlerbehandlung. Weiters müssen auch andere Dinge wie zyklische Abhängigkeiten überprüft werden (siehe Abschnitt \ref{implementierung_zyklenueberpruefung}). Die Aus\-führ\-ung der Statements erfolgt beim Ausfüllen der Daten in ein Formular, wenn es Berechnungen durch Formeln enthält. Hier ist vor allem die Reihenfolge der Berechnungen interessant, da natürlich Felder, von denen ein anderes Feld abhängt, zuerst berechnet werden müssen (siehe Abschnitt \ref{implementierung_integration_reihenfolge}).


\section{Sicherheit und Stabilität}

Fehlerhafte Statements - Ausschluß von Endlosschleifen

Endlosrekursion - StackOverflowError

Statische Analyse vor jeder Aus\-führ\-ung oder Beim Speichern

Probleme/Sicherheitsbedenken durch editieren der Scripts in der Datenbank bzw. Import beschädigter Formulare

Evaluierung zu null????




\section{Än\-der\-ung\-en im Datenmodell}

Die individuell zusammengestllten Formulare werden durch drei Klassen abgebildet, die per JPA in der Datenbank persistiert werden. Wie in Abbildung \ref{abb_uml_datenmodell} ersichtlich, besteht ein Formular aus einer oder mehreren Gruppen von Formelementen. Eine Gruppe wiederum besteht aus den Klassen der einzelnen Formelemente (Textfield, Checkbox etc.). Da die Variablen, Formeln und Constraints für die einzelnen Formularelemente ebenfalls ge-spei-chert werden müssen, wurde eine neue Klasse \texttt{DslAttribute} hinzugefügt.

\begin{figure}[ht]
\begin{center}
 
\includegraphics[scale=0.7]{figures/uml_datenmodell_neu}
\end{center}

\caption{Datenmodell der Formulare}
\label{abb_uml_datenmodell}
\end{figure}

Jedes Formelement enthält eine boolesche Variable namens dslEnabled. Dieses Flag bestimmt, ob der Untertyp der Klasse Formelement zum Berechnen durch die FXL geeignet ist. Wenn ja, bekommt jedes Objekt dieses Formelements einen Verweis auf ein Objekt vom Typ DslAttribute. Die Klasse DslAttribute selbst enthält alle Informationen, die ein DSL-enabled Feld benötigt: den Variablennamen, eine Formel und einen Constraint. 

Es gibt zwei Möglichkeiten, die neue Klasse in das Datenmodell einzubinden: Als One-to-one Relation mit einem Foreign Key in einer der Klassen, oder als Embedded Object in der Klasse \texttt{Formelement}. In der ersten Variante wird eine eigene Tabelle für die Klasse \texttt{DslAttribute} erstellt. Über ein Id-Feld, welches zur Klasse hinzugefügt werden muss, wird die neue Tabelle über ein Foreign-Key Feld mit einem Formelement verbunden.

Die zweite Variante, die auch in der Implementierung verwendet wird, verändert lediglich die bestehende Tabelle der Formelemente. Es werden drei Spalten hinzugefügt, die die Attribute der Klasse \texttt{DslAttribute} repräsentieren. Der Vorteil dieser Methode ist, dass eine Tabelle weniger erstellt wird und somit ein Join bei der Abfrage eingespart wird.

\begin{lstlisting}[float = htbp,caption={Embedded DslAttribute },label=listing_dslattribute]
@Embeddable
public class DslAttribute {

  private FormElement formElement;
  
  @Column
  private String variableName;

  @Column
  private String formula;

  @Column
  private String constraintFormula;

  ...
}
\end{lstlisting}


Listing \ref{listing_dslattribute} enthält einen Auszug aus der neu erstellten Klasse, ohne Getter bzw. Setter und den Methoden toSting(), hashCode() und equals().

\section{DSL Eingabe}
\label{implementierung_dsl_eingabe}

Die Eingabe des Variablennamens, der Formel und des Constraints erfolgt in der Eingabemaske des jeweiligen Formelements (siehe Abbildung \ref{abb_screenshot_spics_eingabe}). Dazu werden, wenn die DSL für den Typ des Formelements aktiviert ist, dem Eingabeformular drei Textfelder hinzugefügt, die per JSF an die Attribute der Klasse DslAttribute gebunden werden.


\begin{figure}[h]
\begin{center}
 \includegraphics[scale=0.5]{figures/screenshot_spics_eingabe_neu}
\end{center}
\caption{Eingabe der DSL im Formulareditor}
\label{abb_screenshot_spics_eingabe}
\end{figure}

Hier wirft sich die Frage auf, ob es sinnvoll ist, die Angabe eines Constraints zu einem Feld, das durch eine Formel berechnet wird, zuzulassen. Meistens ist der Wert eines durch eine Formel berechneten Feldes ohnehin von anderen Feldern abhängig. Es erscheint also sinnvoll, die Validierung durch Constraints dort durchzuführen, um der Formel gültige Werte zu liefern. Andererseits muss eine Formel nicht von anderen Feldern abhängig sein\footnote{{Es kann durch Funktionen beispielsweise eine Formel erstellt werden, die einen Datumswert aus dem aktuellen Datum berechnet. Hier könnte ein Constraint der, z.B. vom Datum eines anderen Feldes abhängt, durchaus Sinn machen.}}. In der ersten Implementierung, die im Zuge dieser Arbeit durchgeführt wird, soll von dieser Einschränkung abgesehen werden. Es bleibt zu evaluieren, inwieweit so eine Einschränkung, auch in Hinsicht auf eine Vereinfachung des User Interfaces und der damit eingehenden Verbesserung der Usability, sinvoll ist.

Beim Speichern wird ein Validator aufgerufen, der jedes der drei Felder gesondert validiert. Im Validator findet die Zyklenüberprüfung und die Überprüfung von Abhängigkeiten statt. Beim Speichern eines Formularelements sind mehrere Aspekte betreffend der DSL zu beachten:

\paragraph{Variablenname}
\begin{itemize}
	\item Der Variablenname darf nur einmal in dem Formular vorkommen.
	\item Der Variablenname darf nur aus Buchstaben und Ziffern bestehen, wobei das erste Zeichen ein Buchstabe sein muss.
	\item Um die Formeln nicht unübersichtlich werden zu lassen, wird die Länge der Variablen auf 15 Zeichen begrenzt.
\end{itemize}

\paragraph{Formel}
\begin{itemize}
	\item Alle Variablen, die in einer Formel verwendet werden, müssen bereits angelegt worden sein. Ein Anlegen der Variablen im Nachhinein soll nicht erlaubt werden, um die Konsistenz zu gewährleisten.
	\item Die Formeln eines Formulars dürfen keine zyklischen Abhängigkeiten enthalten (siehe Abschnitt \ref{implementierung_zyklenueberpruefung}).
	\item Die Formel darf sich nicht selbst referenzieren (also ein Verweis auf den Variablennamen des eigenen Feldes), da dies wiederum einer zyklischen Abhängigkeit entsprechen würde.
	\item Der Rückgabetyp der Formel muss dem Datentyp des Formelements entsprechen, da das Ergebnis der Formel sonst nicht zugewiesen werden kann.
\end{itemize}

\paragraph{Constraint}
\begin{itemize}
	\item Ein Constraint muss als Rückgabetyp Boolean haben, da er über die Gültigkeit der Eingaben entscheidet.
	\item Im Gegensatz zur Formel, darf sich ein Constraint auf den Wert des eigenen Feldes, sprich den eigenen Variablennamen beziehen. Dies ist not\-wen\-dig, um das eigene Feld zu Validieren.
	
\end{itemize}

\section{Zyklenüberprüfung}
\label{implementierung_zyklenueberpruefung}

Ein essentieller Bestandteil der Validierung ist die Überprüfung auf zyklische Abhängigkeiten innerhalb eines Formulars. Diese Abhängigkeiten können als gerichteter Graph interpretiert werden: Enthält ein Formelement eine Formel mit Variablen, so sind alle Formelemente, die durch diese Variablen repräsentiert werden Nachfolger des Formularelements. Ist umgekehrt ein Formelement durch einen Variablennamen markiert, sind alle Formelemente, die eine Formel mit diesem Variablennamen enthalten, Vorgänger des Formelements. Abbildung \ref{abb_cycle_bmi_tree} zeigt den Graphen für das Beispielformular zur BMI-Berechnung.

\begin{figure}[ht]
\begin{center}
 \includegraphics[scale=0.5]{figures/cycle_bmi_tree}
\end{center}
\caption{Baum der Abhängigkeiten im BMI-Formular}
\label{abb_cycle_bmi_tree}
\end{figure}

Die Interpretation der Abhängigkeiten als gerichteter Graph hilft bei der Überprüfung auf Zyklen. Ein Zyklus (oder Kreis) ist ein Weg von einem Knoten zu sich selbst\cite{Schl08}. Es gibt zwei grund\-sätz\-liche Algorithmen die sich für die Überprüfung eignen: die Überprügung durch Tiefensuche oder Breitensuche.

Die Tiefensuche ist ein Algorithmus, bei dem der Graph rekursiv abgearbeitet wird. Die bereits besuchten Knoten werden in einer globalen Liste abge-spei-chert. Wird ein Knoten besucht, der sich bereits in der Liste befindet, so liegt ein Zyklus vor. Durch Erweiterungen (merken der Wege) können so alle Zyklen eines Graphen identifiziert werden.

Ein weniger mächtiger Algorithmus ist die Breitensuche. Diese ist ausreichend, wenn nur festgestellt werden soll, ob sich ein Knoten in einem Zyklus befindet oder nicht. Im folgenden wird der Algorithmus zur Zyklenüberprüfung in der Implementierung (mittels Breitensuche) erläutert.

Die Eingabe ist das aktuelle Feld, das überprüft werden soll. Wenn keine Zyklen vorliegen, terminiert der Algorithmus. Wenn ein Zyklus gefunden wird, wird eine Exception geworfen. Ein Feld ist von einem anderen abhängig, wenn der Variablenname des Feldes in der Formel des anderen Feldes vorkommt.

\begin{enumerate}
  \item Überprüfe das aktuelle Feld ob ein Variablenname und eine Formel vorhanden ist. Ist dies nicht der Fall kann es nicht Teil eines Zyklus sein und die Überprüfung wird ohne Fehler beendet.
  
  \item Initialisiere die Menge \texttt{visited}. Sie enthält jene Felder, die bereits besucht wurden. 

  \item Füge den Variablennamen des aktuellen Feldes hinzu.

  \item Initialisiere die Menge \texttt{next}. Sie enthält jene Felder, die von den aktuell überprüften Feldern abhängen.

  \item Füge alle Variablen der Formel des aktuellen Feldes hinzu.

  \item Solange \texttt{next} nicht leer ist:

    \begin{enumerate}
      \item Wenn \texttt{visited} $\cup$ \texttt{next} $\neq$ \{\} : Es liegt ein Zyklus vor. Wirf Exception.
      \item Füge alle Elemente aus \texttt{next} zu \texttt{visited} hinzu.
      \item \texttt{next} = alle Nachfolger der Felder aus \texttt{next}.
    \end{enumerate}
\end{enumerate}


\section{Dateneingabe}
\label{implementierung_daten_eingabe}

Die Eingabe der medizinischen Daten in die Dokumentationssoftware ist der eigentliche Einsatzort der DSL. Hier werden die Formeln anhand der eingegebenen Daten berechnet und die Constraints überprüft. Vor der Implementierung werden allerdings einige Überlegungen benötigt, wie die Berechnung und Validierung der Werte durch die Aus\-führ\-ung der Formeln und Constraints am Besten umgesetzt wird.


\subsection{Aus\-führ\-ungszeitpunkt}

Ein Aspekt der Dateneingabe ist die Frage, wann die Formeln des Formulars ausgeführt werden. Im Beispiel des BMI-Rechners muss der BMI aus den Werten der Felder Kör\-per\-grö\-ße und Körpergewicht ausgerechnet werden. Es gibt mehrere Optionen, wann die Berechnung ausgelöst werden soll.

\begin{itemize}
	\item Beim Abspeichern eines Formulars: Die Felder, die durch Formeln berechnet werden, bleiben leer. Erst beim Speichern eines Formulars werden die abhängigen Felder berechnet. Wenn ein Fehler auftritt, wird der Benutzer zum Formular zurückgeleitet und aufgefordert, die Fehler zu beheben. Dies ist die ``sicherste'' Lösung, weil die asynchrone Kommunikation mit dem Server wegfällt. Diese Lösung ist allerdings nicht akzeptabel, da die Felder schon während des Ausfüllens berechnet werden sollen.

	\item In `Echtzeit' bei der Dateneingabe: Die Formeln werden nach der Eingabe eines Werts in ein Formularelement über einen Ajax-Validator berechnet. Der Eingabewert des Feldes wird nach der Eingabe\footnote{In der Implementierung wird, abhängig vom Typ des Eingabefeldes, entweder das \texttt{onblur}- oder \texttt{onchange}-Event des Formelements verwendet.} an den Server gesendet. Dort werden jene Felder, die vom eingegebenen Wert abhängen, berechnet und an den Client zurückgegeben, der die berechneten Werte in die ent\-sprech\-enden Felder einfügt.

	Die Möglichkeit der zusätzlichen wiederholten Berechnung nach dem Speichern bleibt allerdings bestehen.

	\item Auf Knopfdruck/Wunsch des Users: Die Werte der Felder werden nicht automatisch nach der Eingabe, sonden nur auf expliziten Wunsch des Users an den Server gesendet und dort berechet. Zu\-sätz\-lich werden die Felder beim Speichern berechnet. Diese Lösung hätte den Vorteil, dass der Benutzer mehr Kontrolle über das Formular hat. Zu\-sätz\-lich werden etwaige Verzögerungen, die bei der `Echtzeit'-Lösung aufteten könnten vermieden.
\end{itemize}

In der Implementierung wurde der zweite Weg gewählt. Die Daten werden in Echtzeit an den Server gesendet und dort ausgewertet. Zu\-sätz\-lich werden die Felder nocheinmal beim Speichern berechnet.

Die Frage der Auswertung stellt sich auch bei jenen Feldern, die mit Constraints belegt sind. Entweder werden die Constraints bereits beim Ausfüllen, oder erst beim Abspeichern evaluiert. Auch hier wurde eine Lösung aus beiden Möglichkeiten implementiert. 


\subsection{Aus\-führ\-ungsort}
\label{integration_ausfuehrungsort}

Webanwendungen bieten heutzutage dank Java\-Script die Möglichkeit, Berechnungen entweder clientseitig durchzuführen, oder das Ergebnis einer Berechnung mittels Ajax-Calls vom Server abzufragen und dann clientseitig weiter zu verarbeiten. Die Frage der client- oder serverseitigen Evaluierung der FXL stellt sich auch bei der Integration in SPICS.

\subsubsection{Clientseitige Evaluierung}

Grundsätzlich wäre eine clientseitige Evaluierung der Formeln möglich, indem man die DSL-Statements in Java\-Script übersetzt und im Browser ausführt. Die Übersetzung von FXL in Java\-Script ist möglich, da Java\-Script alle Sprachkonzepte, die FXL verwendet unterstützt\footnote{Tatsächlich das gleiche Verhalten wie Java\-Script zu erzeugen ist nicht ganz so einfach, da Java\-Script bei manchen Operatoren ein anderes Verhalten an den Tag legt.}. Wenn eine Formel den Wert eines Feldes verändert, so wird dieser neue Wert im ent\-sprech\-enden Formularfeld gesetzt. Erst beim Speichern eines Formulars werden die berechneten und eingegebenen Werte an den Server gesendet, wo diese validiert und persistiert werden (Abbildung \ref{abb_uml_seq_client}).

\begin{figure}[th]
\begin{center}
\includegraphics[scale=0.55]{figures/uml_seq_client_neu}
\end{center}

\caption{Clientseitige Lösung}
\label{abb_uml_seq_client}
\end{figure}


Ein Schwachpunkt dieser Lösung ist, dass die in Java programmierten Funktionen, die in FXL verwendet werden können, nicht so einfach in Java\-Script übersetzt werden können wie die DSL-Statements selbst. Es ist also eine eigens für den Browser entwickelte API not\-wen\-dig, die alle Funktionen enthält, die der DSL auch am Server zur Verfügung stehen. Die Variablen können einfach aus den betreffenden Formularfeldern ausgelesen werden.

Eine Möglichkeit, den Overhead einer zusätzlichen clientseitigen Implementierung der Funktionen zu beseitigen, wäre die Weiterleitung der Funktionsaufrufe an den Server per Ajax-Call (Abbildung \ref{abb_uml_seq_client_hybrid}). Ein Bean am Applicationserver ruft die Methode in der Java-Implementierung auf und sendet den Rückgabewert zurück an den Browser. Der Nachteil ist, dass der Performance-Vorteil der clientseitigen Implementierung bei mehreren Funktionsaufrufen verloren geht.

\begin{figure}[h]
\begin{center}
\includegraphics[scale=0.55]{figures/uml_seq_client_hybrid_neu}
\end{center}

\caption{Client- und serverseitige Hybridlösung}
\label{abb_uml_seq_client_hybrid}
\end{figure}

Ein weiterer Schwachpunkt der clientseitigen Lösung ist die Manipulierbarkeit von Skripten im Browser. Durch ent\-sprech\-ende Tools wie Firebug\footnote{http://getfirebug.com/}, können Nutzer das DOM und die Skripte manipulieren, und so eine korrekte Aus\-führ\-ung der DSL verhindern. Ein weiteres Defizit ist, dass die präzisen Fehlermeldungen, die für FXL definiert wurden, nun nicht nutzbar sind, da diese nicht in Java\-Script transferiert werden können.

\subsubsection{Serverseitige Evaluierung}

Die Alternative zur Evaluierung der Statements im Browser ist die Aus\-führ\-ung am Server (Abbildung \ref{abb_uml_seq_server}).

\begin{figure}[ht]
\includegraphics[scale=0.55]{figures/uml_seq_server_neu}
\caption{Aus\-führ\-ung am Server}
\label{abb_uml_seq_server}
\end{figure}

Das \texttt{onblur}- bzw. \texttt{onchange}-Event des Formularelements löst per JSF Ajax\-Support die Validierung des Formularelements am Applicationserver aus. Der eigens für die DSL implementierte DslValidator ruft die Komponente DslExecuter auf, die die DSL-Statements in korrekter hierarchischer Reihenfolge der Abhängigkeiten ausführt (vgl. Algorithmus in Abschnitt \ref{implementierung_integration_reihenfolge}). Für jede FXL-Formel wird ein Interpreter erstellt, der die Formel evaluiert und dabei auf die Implementierung des \texttt{VariableProvider} Interfaces und den Functionmanager zugreift. Zu\-sätz\-lich zu dem Feld, das das Event ausgelöst hat, werden auch die abhängigen Formularelemente neu gerendert.

Tritt bei der Aus\-führ\-ung der DSL-Statements ein Fehler auf, so wird eine ent\-sprech\-ende \texttt{FXLException} geworfen und über die Komponenten bis zum DslValidator weitergeleitet, wo diese abgefangen wird. Mithilfe des Fehlercodes wird eine FacesMessage erstellt und an den Context der Applikation übergeben. Diese Fehlermeldung wird im Browser beim betreffenden Formelement angezeigt.

Die serverseitige Lösung ist die einfachere Lösung, da die Aus\-führ\-ung einfach mittels FXL funkioniert. Auch das Fehlerhandling erfolgt über die dafür vorgesehenen Exceptions, die dem Benutzer detailliertes Feedback geben. Ob die Verzögerung durch den Ajax-Call eine unzumutbare Verzögerung für den Benutzer darstellt, kann in einer weiteren Arbeit evaluiert werden. 


\subsection{Aus\-führ\-ungsreihenfolge}
\label{implementierung_integration_reihenfolge}

Die Aus\-führ\-ungsreihenfolge ist essentiell für die korrekte Auswertung der Formeln eines Formulars. Bevor ein Feld zur Berechnung herangezogen wird, muss sichergestellt werden, dass dieses selbst bereits berechnet wurde, wenn es eine Formel enthält. Wie bereits in Abschnitt \ref{implementierung_zyklenueberpruefung} erläutert, können die Abhängigkeiten in einem Formular als gerichteter zyklenfreier Graph gesehen werden. Wird ein Formularelement in der Formel eines anderen Elements verwendet, so kann dies als eine gerichtete Kante gedeutet werden.

Der folgende Algorithmus geht davon aus, dass keine Zyklen vorliegen. Die Zyklenfreiheit muss bereits beim Erstellen des Formulars sichergestellt werden (Abschnitt \ref{implementierung_zyklenueberpruefung}).

\begin{enumerate}
	\item Suche alle Felder, die einen Variablennamen, aber keine Formel beinhalten. Schreibe alle Werte der Felder in die Map \texttt{calculated<String, 			Result>}, wobei der Key der Variablenname ist und der Value der Wert des Feldes in einem Result Objekt.
	\item Erstelle eine Liste \texttt{formulas} aller Felder, die Formeln besitzen, sprich berechnet werden müssen.
	\item Solange \texttt{formulas} nicht leer ist, für jede Formel:
	
	\begin{enumerate}
		\item Hole alle Variablen aus der Tokenfolge der Formel.
		\item Wenn alle Variablen im KeySet von \texttt{calculated} sind:
		\begin{enumerate}
			\item Berechne die Formel.
			\item Wenn das Formelement der Formel einen Variablennamen hat, schreibe diesen mit dem berechneten Wert in \texttt{calculated}.
			\item Lösche die Formel aus \texttt{formulas}.
		\end{enumerate}	
		\item Wenn nicht alle Variablen bereits berechnet sind, ignoriere die Formel.
	\end{enumerate}	
	
\end{enumerate}

Der Algorithmus ist recht einfach: Zuerst werden alle Felder als berechnet angesehen, die selbst keine Formel beinhalten. Danach werden jene Felder berechnet, deren Formeln nur Variablen beinhalten, die auf bereits berechnete Felder verweisen. Durch die zugesicherte Zyklenfreiheit kann dieser Vorgang wiederholt werden, bis alle Felder berechnet sind.

Im Beispiel des BMI-Rechners (Abbildung \ref{abb_cycle_bmi_tree}) werden zuerst die Felder \emph{Körpergrösse} \& \emph{Körpergewicht} als berechnet markiert, da diese nicht von anderen Feldern abhängig sind. Danach werden die Felder mit Formeln, hier \emph{Adipositas} bzw. \emph{BMI}, in eine Liste ge-spei-chert, die in einer Schleife abgearbeitet wird. Wird zuerst versucht, das \emph{Adipositas}-Feld zu berechnen, wird festgestellt, dass das Feld \emph{BMI} noch nicht berechnet wurde und der Versuch verworfen. Danach wird das Feld \emph{BMI} aus den Feldern \emph{Körpergrösse} \& \emph{Körpergewicht} berechnet und aus der Liste entfernt. Beim nächsten Schleifendurchlauf kann nun die Formel der \emph{Adipositas}-Checkbox ausgewertet und aus der Liste gelöscht werden.

Bei der Aus\-führ\-ung wirft sich die Frage auf, wie vorgegangen werden soll, wenn ein Eingabefeld mit Variablenname noch nicht ausgefüllt wurde. In diesem Fall liefert die \texttt{lookup()}-Methode des \texttt{VariableProvider} ein \texttt{Result}-Objekt vom richtigen Datentyp, aber mit Wert \texttt{null} zurück. Die ganze Formel wird, wie im Abschnitt \ref{design_interpreter} beschrieben, zu einem \texttt{Result}-Objekt mit Wert \texttt{null} ausgewertet. Ist ein Feld also noch leer, werden auch die abhängigen Felder nicht mit einem Wert belegt. Wichtig ist jedoch, dass die Aus\-führ\-ung nicht unterbrochen wird und alle Felder, die nicht von einem nicht ausgewerteten Feld abhängen können, berechet werden.



\section{Fehlerbehandlung}
\label{integration_fehlerbehandlung}

Durch die sehr detailierten Fehlermeldungen der DSL-Implementierung ist es möglich, dem Benutzer ein genaues Feedback zu geben, wenn Probleme auftreten. Zu\-sätz\-lich zu den vorgegebenen Fehlern ist es sinnvoll, auch implementierungsspezifische Fehlermeldungen zu definieren.

Wenn Exceptions bei der statischen Analyse oder der Aus\-führ\-ung eines Statements auftreten, so müssen diese intelligent weitergeleitet und an der richtigen Stelle im System abgefangen werden, um einen reibungslosen Ablauf des Workflows zu ermöglichen, aber vor allem den ordnungsgemäßen Betrieb des Systems nicht zu gefährden.

\subsection{Implementierungsfehler}

In der Dokumentationssoftware können Fehler an zwei Stellen auftreten: Bei der Modellierung eines Formulars und bei der Aus\-führ\-ung der Formeln und Constraints beim Ausfüllen eines Formulars. Jene Fehler, die beim Ausfüllen auftreten, wie etwa eine Division durch null, werden bereits von den in Abschnitt \ref{section_design_fehlerbehandlung} definierten Implementierungsfehlern abgedeckt. Wenn eine ent\-sprech\-ende \texttt{FXLException} geworfen wird, wird diese am Server abgefangen. Abhand des Fehlercodes wird die ent\-sprech\-ende internationalisierte Nachricht erstellt (siehe unten). 


\begin{table}
\begin{tabular}[ht]{|c | p{6cm} | p{2cm}| p{3cm} |}
	\hline
	\textbf{Code} & \textbf{Beschreibung} & \textbf{Beispiel} & \textbf{Argumente}\\
	\hline
  	\hline
  	4xx  & \multicolumn{3}{|l|}{\textsc{Implementierungsfehler}} \\
  	\hline
  	400  & Allgemeiner/unbekannter Implementierungsfehler &  & \\
  	\hline
  	411  & Ungültiger Variablenname & & \\
  	\hline
  	412  & Variablenname zu lang & & \\
  	\hline
  	413  & Variablenname bereits vorhanden & &\\
  	\hline
  	414  & Die Formel erzeugt zyklische Abhängigkeiten & & \\
  	\hline
  	415  & Die Formel enthält unbekannte Variablen & & \\
  	\hline
  	416  & Rückgabetyp der Formel stimmt nicht mit Datentyp des Formelements überein & & \\
  	\hline
  	417  & Rückgabetype eines Constraints ist nicht Boolean & & \\
  	\hline
  	4xx  & TODO: Tabelle schön machen & & \\
  	\hline
\end{tabular}
\caption{Fehlercodes für Implementierungsfehler}
\end{table}

Neu und spezifisch für den Einsatzbereich der DSL sind jene Fehler, die beim Erstellen der Formulare auftreten. Diese decken vor allem Fehler des Formats von Variablen, der logischen Struktur der Abhängigkeiten und der Rückgabetypen ab.



\subsection{Internationalisierung}

Die Internationalisierung der Fehlermeldungen erfolgt mit ``Boardmitteln'' von Seam\footnote{ vgl. http://docs.jboss.org/seam/latest/reference/en-US/html/i18n.html, 21.12.2011}. Die Labels werden in Properties-Files für alle unterstützten Sprachen ge-spei-chert (Listing \ref{listing_error_de} bzw. \ref{listing_error_en}).

\begin{lstlisting}[float = htbp,caption={Deutsche Fehlermeldungen },label=listing_error_de]
error.dsl.202=Ungueltige Kombination von Typen fuer Operator {0}: {1}, {2}
error.dsl.211=Variable nicht gefunden: {0}
error.dsl.212=Variable mehrfach definiert: {0}
\end{lstlisting}

\begin{lstlisting}[float = htbp,caption={Englische Fehlermeldungen},label=listing_error_en]
error.dsl.202=Invalid type combination for operator {0}: {1}, {2}
error.dsl.211=Variable not found: {0}
error.dsl.212=Variable defined more than once: {0}
\end{lstlisting}

Zur Identifizierung der Labels wird der Fehlercode der Exception verwendet. Um das Feedback duch fehlerspezifische Informationen anzureichern, werden die Argumente der Fehlermeldungen, wie in Listing \ref{listing_error_de} bzw. \ref{listing_error_en} ersichtlich verwendet.






\part{Ergebnis}
\label{part_ergebnis}



\chapter{Zusammenfassung}
\label{chapter_zusammenfassung}

Das Thema dieser Arbeit entstand aus der Anforderung, Formulare in einer Webanwendung semantisch dahingehend anzureichern, dass flexible Berechnungen und Validierungen formelementübergreifend vorgenommen werden können. Im Rahmen der Arbeit wurde eine einfache Ausdruckssprache entwickelt und in ein medizinisches Studiensystem integriert.

Der erste Teil der Arbeit beschäftigt sich mit den technischen und theoretischen Grundlagen, die zum Verständnis des Textes notwendig sind. Zunächst wurde der Begriff der Domain Specific Language(DSL) diskutiert, eine Einteilung vorgenommen und Beispiele für bekannte DSLs vorgestellt. Danach wurden die theoretischen Grundlagen erarbeitet und eine Übersicht über Werkzeuge zum Entwurf von Sprachen geboten. Den Abschluss des ersten Teils bildet eine Übersicht über wissenschaftliche Arbeiten im Bereich der Modellierung von Webformularen.

Im zweiten Teil wird der Entwicklungsprozess der Form Expression Language (FXL), der sich in die Phasen Entscheidung, Analyse, Entwurf, Implementierung und Test gliedert, beschrieben. Zuerst wurde eine Entscheidung für eine eigene externe DSL getroffen und begründet. Danach wurden die Anforderungen an die Sprache analysiert und das Design der Sprache in Hinsicht auf die notwendigen Features wie Datentypen, Operatoren, Funktionen und Fehlerbehandlung, sowie Syntax und Semantik entwickelt. Danach wurden die Implementierungsdetails der FXL dargelegt. Den Abschluss des Teils bildet eine Abhandlung über das Testen der erstellten Software.

Der dritte Teil behandelt die Integration der Sprache in ein existierendes medizinisches Studiensystem. Zuerst wurde das Studiensystem SPICS kurz vorgestellt und die Architektur sowie die verwendeten Technologien beschrieben. Danach wurden die verschiedenen Aspekte der Integration der Sprache in SPICS veranschaulicht.

Der vierte Teil gibt einen Ausblick auf weitere Themen und Möglichkeiten, die im Zuge der Arbeit aufgetan haben und bietet eine zusammenfassende Übersicht über die Themen der Arbeit.

Ein Schwerpunkt der Arbeit war die praktische Implementierung der FXL, sowie die Integration in das medizinische Studiensystem. Die Arbeitsschritte entsprechen im Grunde den in der Arbeit dargestellten Schritten, mit dem Unterschied, dass die praktische Umsetzung natürlich ein iterativer Prozess war. Zuerst wurden die Anforderungen und das Zielsystem analysiert und eine Übersicht über das wissenschaftliche Feld erstellt. Danach wurden diverse Tools und Möglichkeiten zur Umsetzung der Sprache in Hinsicht auf die Anforderungen evaluiert. Im nächsten Schritt erfolgte die testgetriebene Implementierung der Fxl, in der die Anforderungen iterativ überarbeitet und immer detailierter spezifiziert wurden. Eine besondere Herausforderung war die Integration in das vorhandene Studiensystem, da auf die Besonderheiten des Systems, wie die Technologien oder das Datenmodell, eingegangen werden musste. Das Ergebnis der praktischen Arbeit ist eine funktionierende Implementierung gemäß der Problemstellung der schriftlichen Arbeit.



\chapter{Ausblick}
\label{chapter_ausblick}

Im Verlauf der Arbeit haben sich einige Themen herauskristallisiert, die sich für weitere Arbeiten, über den Umfang dieser Diplomarbeit hinaus, anbieten. In der Implementierung wurde die Integration der Form Expression Language in SPICS nur für drei verschiedene Formelemente vorgenommen, die ausreichen um die vier unterstützten Datentypen zu repräsentieren: Textfelder für die numerischen Typen, sowie Checkboxen für boolesche Werte und Datepicker für Datumswerte. Für die Verwendung von weiteren Formelementen wie Listen, Drop-Downs und Radiobuttons können Überlegungen angestellt werden, wie Werte dieser Elemente mit der FXL verarbeitet werden können.

Referenzierungen in den Formeln der FXL lassen sich momentan nur auf Formularfelder innerhalb eines Formulars vornehmen. Da für einen Patienten in SPICS unterschiedliche Formulare angelegt werden können, kann die Referenzierung von Feldern aus anderen Formularen eine weitere Anforderung darstellen. Dafür kann die Implementierung der Variable Provider dahingehend geändert werden, dass ein Lookup von Variablen auch Werte bereits persistierter Formulare berücksichtigt. 

Auch die Form Expression Language selbst lässt Raum für Erweiterungen in mancherlei Hinsicht. Für eine formularübergreifende Referenzierung könnte etwa die Syntax für Variablennamen modifiziert werden. Eine weitere mögliche Erweiterung stellt die Einführung eines Datentyps für Zeichenketten, mit entsprechen Auswirkungen auf die Syntax der Sprache sowie die Semantik der Operatoren, dar. Eine zusätzliche Möglickeit stellt die Definition weiterer Operatoren dar. In der Implementierung dieser Arbeit kann der ganzzahlige Rest einer Division (Modulo) nur durch eine eigene Funktion berechnet werden. Für diese Berechnung könnte, wie in anderen Sprachen üblich, der Operater \texttt{\%} eingeführt werden. Auch eigene logische Operatoren für exklusives Oder und Implikation sind denkbar. Diese Logischen Operatoren müssen momentan durch die vorhandenen Grundoperatoren ausgedrückt werden.

Ein weiteres Thema sind die nichtfunktionalen Eigenschaften der Integration im Studiensystem. Vor allem in Hinsicht Usability kann die Implementierung einer Evaluierung unterzogen werden. Die Akzeptanz beim Endanwender kann durch User-Tests bestätigt oder widerlegt werden. Auch die Eingabe der Statements lässt Raum für Verbesserungen im User Interface. Eine Möglichkeit wäre beispielsweise Syntax Highlighting oder Code Completion anzubieten, um die Eingabe angenehmer zu gestalten und Fehler zu vermeiden. Auch in Hinblick auf die Performance der Sprache, sowie die Integration in das Studiensystem, können Untersuchungen vorgenommen werden. Vor allem die Möglichkeiten einer teilweisen clientseitigen Berechnung, wie in Abschnitt \ref{integration_ausfuehrungsort} dargelegt, können Thema einer weiteren Untersuchung sein.



%\chapter{Terms and abbreviations}
%


AST - Abstract Sysntax Tree, CST Concrete Syntax Tree, GPL - General Purpose Language , Syntax, Semantik, Statement

Constraint - Bedingung



\begin{flushleft}
\bibliographystyle{plain}
\bibliography{biblio}
\end{flushleft}


\listoftables
\listoffigures
\lstlistoflistings

\appendix
% include appendexes here
\chapter{Grammatik}

\begin{lstlisting}[caption={Grammatik der FXL},label=listing_fxl_grammar]
grammar FXL;

options {
  language = Java;
  output   = AST;
}

tokens {
  OR  = 'OR';
  AND = 'AND';
  NOT = 'NOT';
  GT  = '>'; //greater then
  GE  = '>='; //greater then or equal
  LT  = '<'; //lower then
  LE  = '<='; //lower then or equal
  EQ  = '=';
  NEQ = '!='; //Not equal
  PLUS = '+';
  MINUS = '-';
  MULTIPLY = '*';
  DIVISION = '/';
  CALL;
}

@header {
package at.ac.tuwien.fxl;
}

@members {
    public void displayRecognitionError(String[] tokenNames,
                                        RecognitionException e) {
        String hdr = getErrorHeader(e);
        String msg = getErrorMessage(e, tokenNames);
        throw new RuntimeException(+e.index+"");
    }
}


@lexer::header {
package at.ac.tuwien.fxl;
}

statement 
  :  ('='|':')! expression 
  ;

expression
	: booleanAndExpression ('OR'^ booleanAndExpression)*
	;

booleanAndExpression
  :
  booleanNotExpression ('AND'^ booleanNotExpression)*
  ;

booleanNotExpression
  :
  ('NOT'^)? booleanAtom
  ;

booleanAtom
  :
  | compareExpression
  ;

compareExpression
  	:
  	commonExpression (('<' | '>' | '=' | '<=' | '>=' | '!=' )^ commonExpression)?
  	;

commonExpression
  :
  multExpr (( '+'^ | '-'^ ) multExpr)*
  ;

multExpr
  :
  atom (('*'|'/')^ atom)*
  | '-'^ atom
  ;

atom
  :
  INTEGER
  | DECIMAL
  | BOOLEAN
  | DATE
  | ID
  | '(' expression ')' -> expression
  | functionCall
  ;

functionCall
  :
  ID '(' arguments ')' -> ^(CALL ID arguments?)
  ;
  
arguments
  :
  (expression) (','! expression)*
  |  WS
  ;

BOOLEAN
  :
  'true'
  | 'false'
  ;

ID
  :
  ('a'..'z' | 'A'..'Z')+ ('a'..'z' | 'A'..'Z' | '0'..'9')*
  ;

INTEGER
  :
  ('0'..'9')+ 
  ;

DECIMAL
  :
  ('0'..'9')+ ('.' ('0'..'9')*)
  ;

DATE
  :
  '!' '0'..'9' '0'..'9' '0'..'9' '0'..'9' '-' '0'..'9' '0'..'9' '-' '0'..'9' '0'..'9'( '-' '0'..'9' '0'..'9' ':''0'..'9' '0'..'9' (':''0'..'9' '0'..'9')?)?
  ;

WS
  :  (' '|'\t' | '\n' | '\r' | '\f')+ { $channel = HIDDEN; };

\end{lstlisting}




\end{document}


